\section{Definition of $\mathcal{M}_{\infty}$}
We start with a basic set of points $\mathcal{M} \subseteq \mathbb{C}$ in the complex plane. 

\begin{definition}[Line]
    \label{def:line}
    A line $l$ through two points $x,y\in\mathbb{C}$ with $x\ne y$ is defined by the set: $$l:=\{\lambda x+(1-\lambda)y\mid\lambda\in\mathbb{R}\}.$$
\end{definition}

\begin{lstlisting}
    structure line where
        (z₁ z₂ : ℂ)

    def line.points (l: line) : Set ℂ:= 
        {(t : ℂ) * l.z₁ + (1-t) * l.z₂ | (t : ℝ)}
\end{lstlisting}

\begin{definition}[Circle]
    \label{def:circle}
    A circle $c$ with center $z\in\mathbb{C}$ and radius $r\in\mathbb{R}_{\ge 0}$ is defined by the set: $$c:=\{z\in\mathbb{C} \mid\|z-c\|=r\}.$$
\end{definition}

\begin{lstlisting}
    structure circle where
        (c : ℂ)
        (r : ℝ)

    def circle.points (c: circle) := Metric.sphere c.c c.r
    noncomputable def circle.points' (c: circle) := 
        (⟨c.c, c.r⟩ : EuclideanGeometry.Sphere ℂ)
\end{lstlisting}


\begin{definition}[Set of lines and circles]
    \label{def:set_of_lines_and_circles}
    $\mathcal{L(M)}$ is the set of all real straight lines defined by two points in $\mathcal{M}$.\\
    And $\mathcal{C(M)}$ is the set of all circles defined by a center in $\mathcal{M}$ and a radius equal to the distance between two points in $\mathcal{M}$.
\end{definition}

\begin{lstlisting}
def L (M:Set ℂ): Set line := {l |∃ z₁ z₂, l = {z₁ := z₁, z₂ := z₂} ∧ z₁ ∈  M ∧ z₂ ∈ M ∧ z₁ ≠ z₂}
def C (M:Set ℂ): Set circle := {c |∃ z r₁ r₂, c = {c:=z, r:=(dist r₁ r₂)} ∧ z ∈ M ∧ r₁ ∈ M ∧ r₂ ∈ M}
\end{lstlisting}

\begin{definition}[Rules to construct a point]
    \label{def:rules_to_constructed_a_point}
    We define operations that can be used to construct new points.
    \begin{enumerate}
        \item $(ILL)$ is the intersection of two different lines in $\mathcal{L(M)}$.
        \item $(ILC)$ is the intersection of a line in $\mathcal{L(M)}$ and a circle in $\mathcal{C(M)}$.
        \item $(ICC)$ is the intersection of two different circles in $\mathcal{C(M)}$.
    \end{enumerate}
    $ICL(\mathcal{M})$ is the set $\mathcal{M}$ combined with all points that can be constructed using the operations $(ILL)$, $(ILC)$ and $(ICC)$.
\end{definition}

\begin{lstlisting}
def ill (M:Set ℂ): Set ℂ := { z  |∃l₁ ∈ L M, ∃ l₂ ∈ L M,  z ∈ l₁.points ∩ l₂.points ∧ l₁.points ≠ l₂.points}
def ilc (M:Set ℂ): Set ℂ := { z  |∃c ∈ C M, ∃ l ∈ L M,  z ∈ c.points ∩ l.points}
def icc (M:Set ℂ): Set ℂ := { z  |∃c₁ ∈ C M, ∃ c₂ ∈ C M,  z ∈ c₁.points ∩ c₂.points ∧ c₁.points' ≠ c₂.points'}    

def ICL_M (M : Set ℂ) : Set ℂ := M ∪ ill M ∪ ilc M ∪ icc M
\end{lstlisting}

\begin{definition}[Set of constructible points]
    \label{def:set_of_constructable_points}
    We define inductively the chain
    \begin{equation*}
        \mathcal{M}_0 \subseteq \mathcal{M}_1 \subseteq \mathcal{M}_2 \subseteq \dots
    \end{equation*}
    with $\mathcal{M}_0 = \mathcal{M}$ and $\mathcal{M}_{n+1} = ICL(\mathcal{M}_n)$.\newline
    And call $\mathcal{M}_{\infty} = \bigcup_{n \in \mathbb{N}} \mathcal{M}_n$ the set of all constructable points.
\end{definition}

\begin{lstlisting}
    def M_I (M : Set ℂ) : ℕ → Set ℂ
        | 0 => M
        | (Nat.succ n) => ICL_M (M_I M n)

    def M_inf (M : Set ℂ) : Set ℂ :=  ⋃ (n : ℕ), M_I M n
\end{lstlisting}

We can now fill in the first blank:
\begin{lstlisting}
    noncomputable def MFinf (M: Set ℂ) : Subfield ℂ where
        carrier := M_inf M
        ...
\end{lstlisting}
