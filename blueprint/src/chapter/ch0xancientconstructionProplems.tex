\chapter{Ancient Construction Proplems}
This chapter will employ the results to demonstrate the impossibility of trisecting the angle and doubling the cube. 
This formalisation is based on the work conducted during my project in Bonn during the Lean Course WiSe 23/24: 
\url{https://github.com/Louis-Le-Grand/LeanCourse23Fork/tree/master/LeanCourse/Project}

\section{Doubling the cube}
The doubling of the cube, also referred to as the Delian problem, represents an ancient geometric problem.
The objective is to construct the edge of a second cube whose volume is double that of the first, using only a ruler and compass, given the edge of a cube.
As previously outlined in the introduction, these can be summarised as follows: 

TODO: Shloud write down the simplification once again or just refer to the introduction?
\begin{problem}
    Let $\mathcal{M} = \{0,1\}$.  
    \begin{equation*}\text{Is the }\sqrt[3]{2} \overset{?}{\in} \mathcal{M}_{\infty}?\end{equation*}
\end{problem}

\begin{lemma}($\sqrt[3]{2}$ is irrational)
    \label{lem:irrational_thirdroot_two}
    \leanok
    \lean{irrational_thirdroot_two}
    The third root of $2$ is an irrational number.
\end{lemma}
\begin{proof}
    The following theorem will be used without proof, as it is already available in MathLib:
    \begin{theorem*}
        For any $x\in(\R\setminus\Z)$ if ther exist an $n\in \N_{>0}$ and $m\in\Z$ such that $m = x^n$, then $x$ is rational. 
    \end{theorem*}
    The fact that $\sqrt[3]{2}^3=2$, allows us to deduce that the only remaining task is to prove that is not an integer. 
    This can be observed in the context of trough two relations.
    \begin{align}
        2^{\frac{1}{3}} &< 2 \\
        2^{\frac{1}{3}} &> 1
    \end{align}
\end{proof}

\begin{lemma}
    $P := X^3 - 2$ is irreducible over $\mathbb{Q}$.
\label{thm:irreducible_over_Q}
\end{lemma}
\begin{proof}
    Since $\mathbb{Q}$ is $a$ subfield of $\mathbb{C} [X]$, we know that
    \begin{equation*}
        X^3 - 2 = (X - \sqrt[3]{2})(X -\zeta_3 \sqrt[3]{2})(X -\zeta_3^2 \sqrt[3]{2})
    \end{equation*}
    Suppose $P$ is Rational, then
    \begin{equation*}
        X^3 - 2 = (X - a)(X^2 + bX + c)\text{, with } a, b, c \in \mathbb{Q}
    \end{equation*}
    In particular it has a zero in $\mathbb{Q}$, so there is a rational number $a$ such that $a^3 = 2$.\newline
    But we know that $\zeta_3 \sqrt[3]{2}$ and $\zeta_3^2 \sqrt[3]{2}$ are not real numberns and $\sqrt[3]{2}$ is not rational \ref{lem:irrational_thirdroot_two}.
    So $P$ is irreducible over $\mathbb{Q}$.
\end{proof}

\begin{theorem}
    The cube can't be doubled using a compass and straightedge.
\end{theorem}
\begin{proof}
    By applying the corollary \ref{cor:Classfication_z_in_M_inf_2m_not}, it is sufficient to demonstrate that no $m\in \N$ exists such that
    $$2^m\overset{?}{=}[\sqrt{3}{2}:\Q({0,1})]\overset{0,1\in\Q}{=} [\sqrt{3}{2}:\Q] = \text{degree}(\mu_{\Q,\sqrt[3]{2}}).$$
    Since $P$ is irreducible over $\mathbb{Q}$ \ref{thm:irreducible_over_Q}, monic and has $\sqrt[3]{2}$ as a zero, we know that $[\mathbb{Q}(\sqrt[3]{2}):\mathbb{Q}] = 3$.
    And since $$ 3  \equiv_2 1 \neq 0  \equiv_2 = 2^m \qquad\forall m \in \N$$
    we can conclude that the cube can't be doubled using a compass and straightedge.
    
\end{proof}

\section{Trisection of an angle}
The trisection of an angle with a compass and ruler can be reduced to the following problem:\\
Let $\mathcal{M} = \{a, b, c\}$ with $a, b, c$ not on a line and $\alpha := \angle (b - a, c - a)$ be the rusulting angle.
Then $\alpha$ can be trisected if and only if, there is a point $d\in \mathcal{M}_{\infty}$ so that $\angle (b - a, d - a) = \alpha/3$. 
The use of a normed set $\mathcal{M} = \{0,1,e^{i\alpha}\}$ leads to the following problem:
\begin{problem}
    Let $\mathcal{M} = \{0,1,\exp(\textbf{i} \alpha)\}$. $$\text{Is }\exp(\textbf{i} \alpha/3) \overset{?}{\in} \mathcal{M}_{\infty}?$$
\end{problem}
In this context, since the numbers zero and one are rational numbers, it can be concluded that $K_0$ is equal to
$$ K_0 = \mathbb{Q}(\mathcal{M}\cup Conj(\mathcal{M})) = \mathbb{Q}(e^{\imath\alpha},\overline{e^{\imath\alpha}}). $$
Given the corollary reference and the fact that $e^{\imath\alpha/3}$ is a zero of $X^3 - e^{\imath\alpha}\in \mathbb{Q}(e^{\imath\alpha},\overline{e^{\imath\alpha}})[X]$, the following is  equivalent: 
\begin{itemize}
    \item $\exp(\textbf{i} \alpha/3) \notin \mathcal{M}_{\infty}$
    \item $\text{degree}(\mu_{e^{\imath\alpha/3},K_0}) = 3$
    \item $X^3 - e^{\imath\alpha}$ is irreducible over $\mathbb{Q}(e^{\imath\alpha},\overline{e^{\imath\alpha}})$
\end{itemize}
The following section will demonstrate that the angel of $\frac{\pi}{3}=60°$  is not trisectable.

\begin{lemma}
    \label{lem:degree_K0_angel}

    The degree of $K_0 = \mathbb{Q}(e^{\imath\frac{\pi}{3}},\overline{e^{\imath\frac{\pi}{3}}})$ is equal  to $2^n$ for $n\in \mathbb{N}$. 
\end{lemma}
\begin{proof}
    Mising proof this will be a fun sunday afternoon task.
    % We know
    % \begin{equation*}
    %     \exp(\textbf{i}x) = \cos(x) + \textbf{i} \sin(x) \quad \forall x \in \mathbb{R}
    % \end{equation*}
    % For $\alpha = \pi / 3$ we get
    % \begin{equation*}
    %     \cos(\alpha) = \frac{1}{2}\qquad \text{and}\qquad \sin(\alpha) = \frac{\sqrt{3}}{2}
    % \end{equation*}
    % Since we know that $\sqrt{r} \in \mathcal{M}_{\infty}$ for $r \in \mathcal{M}_n$
    % we see that $\exp(\textbf{i} \alpha) \in \mathcal{M}_{\infty}$ for $\mathcal{M} = \{0,1\}$. \newline
    % So we will work with $K_0 = \mathbb{Q}$.
\end{proof}

\begin{lemma}
    \label{lem:pi_third_not_in_M_inf}
    \leanok
    \lean{pi_third_not_in_M_inf}
    The angle $\pi / 3 = 60^{\circ}$ can't be trisected using a compass and straightedge.
\end{lemma}
\begin{proof}
    By utilising the aforementioned lemma \ref{lem:degree_K0_angel} to apply the corresponding corollary \ref{cor:Classfication_z_in_M_inf_2m_rat}, we can narrow our focus to the degree over $\Q$.
    Now we use the fact that if $x\in \mathcal{M}_{\infty}$, then $x.real, x.imag \in \mathcal{M}_{\infty}$ \ref{lem:M_inf_properties}. So we focus on $\cos(\alpha/3)$, witch the real part and is zero of
    \begin{equation*}
        f := 8 X^3 - 6 X - 1 \in \mathbb{Q}[X]
    \end{equation*}
    Suppose $f$ is reducible over $\mathbb{Q}$, then $f$ has a rational zero $a$, since $f$ is of degree $3$. According to the rational root theorem, a root rational root of $f$ is of the form $\pm \frac{p}{q}$ with $p$ a factor of the constant term and $q$ a factor of the leading coefficient. So the only possible rational zeros of $f$ are
     \begin{equation*}
        \{ \pm 1, \pm \frac{1}{2}, \pm \frac{1}{4}, \pm \frac{1}{8} \}.
     \end{equation*}
     One can check that none of these numbers is a zero of $f$.
     So $f$ is irreducible over $\mathbb{Q}$ and $\cos(\alpha/3) \notin \mathcal{M}_{\infty}$.
     Therefore
        \begin{equation*}
            \exp(\textbf{i} \alpha/3) \notin \mathcal{M}_{\infty}
        \end{equation*}
    So the angle $\pi / 3 = 60^{\circ}$ can't be trisected using a compass and straightedge.
\end{proof}

\begin{theorem}
    \label{thm:Angle_not_Trisectable}
    \leanok
    \lean{Angle_not_Trisectable}
    A general angle can't be trisected using a compass and straightedge.
\end{theorem}
\begin{proof}
    \leanok
    Employ the previous lemma with the angle $\pi / 3$.
\end{proof}