\chapter[Introduction]{Introduction in the topic of the Bachelor Thesis}

This paper formalises the proof of "The Impossibility of Trisecting the Angle and Doubling the Cube" in ruler and compass construction. 
The Github repository with the code and bluebrint can be found here : \url{https://github.com/Louis-Le-Grand/Formalisation-of-constructable-numbers}
\section{Construction Problems}
The straightedge and compass constructions were developed by the ancient Greeks. 
It consists of an initial set $\mathcal{M}$ of constructed points, a ruler that has no measurements and can draw indefinite lines through at least two existing points,
and a compass, the centre of which is an already constructed point and the radius of which is the distance between two already constructed points. 
To construct new points, take the intersection of two lines, two circles or a line and a circle. The set of all possible intersections is called $\mathcal{M}_{\infty}$. 
In order to construct a line or a circle, it is necessary that our initial set contain at least two points. Consequently, we can normalise our set and assume that $0, 1 \in \mathcal{M}$.

We may now proceed to define the problem of doubling a cube. 
The volume of a cube is equal to the cube of the length of an edge, which may be expressed as $a^3$, where $a$ is the length of an edge. 
Therefore, a cube with a doubled volume, $2\cdot a^3$, has an edge length of the cube root of two times the length of the original edge. 
If we now take the cube with a length of one, the problem is as follows:
\begin{problem}[Doubling the Cube]
    Given the set $\mathcal{M} = \{0, 1\}$, can we construct the point $\sqrt[3]{2}$, i.e. is $\sqrt[3]{2} \in \mathcal{M}_{\infty}$?
\end{problem}

A similar approach allows the problem of trisecting angles to be simplified. 
If an angle is defined by the points $0$, $1$ and $e^{\imath\cdot\alpha}$, the problem can be described as follows:

\begin{problem}[Trisecting the Angle]
    Given the set $\mathcal{M} = \{0, 1, e^{\imath\cdot\alpha}\}$, can we construct the point $e^{\imath\cdot\frac{\alpha}{3}}$, i.e. is $e^{\imath\cdot\frac{\alpha}{3}} \in \mathcal{M}_{\infty}$?
\end{problem}

\section{Motivation}
The subject of this formalisation is the eighth theorem of Freek Wiedijk's list of "100 Theromes", entitled "The Impossibility of Trisecting the Angle and Doubling the Cube" \cite{Wiedijk}. 
This is a list of theorems based on an online list from 1999 of the 100 most significant theorems in mathematics \cite{Abad_Abad}, which is used for comparative purposes with respect to different theorem provers. 
The list provides a concise overview of the most important theorem provers, showcases fields in which theorems can be used, and also presents some smaller programs that have formalised theorems that had not yet been formalised in any other environment.
 Notably, the aforementioned theorem had not yet been formalised in Lean \cite{Lean_Community}.

\section{Lean, Theorem Prover and Mathlib}
Lean is a functional programming language that can be utilised as an interactive theorem prover. 
The Lean Project was initiated by Leonardo de Moura at Microsoft Research Redmond in 2013 and represents a long-term research endeavour, published under the Apache 2.0 licence.

In order to verify a theorem, it is necessary to find proof. 
Computers can be of assistance in two distinct ways: 
firstly, through interactive theorem proving, which verifies the correctness of a proof step by step; 
and secondly, through automated theorem proving, which attempts to find a proof for a given statement. 
Lean represents a hybrid of these two approaches. 
It is an interactive theorem prover, but it incorporates automated tools and methods to assist in the construction of a fully specified axiomatic proof.\cite{Avigad_deMoura_Kong_Ullrich}


Mathlib is a library of mathematical content for the Lean programming language. It is a community-developed project, created by a large number of contributors and covering a substantial breadth of mathematical topics. To ascertain which areas are encompassed by MathLib, one may refer to the following link: \url{https://leanprover-community.github.io/mathlib-overview.html}. 