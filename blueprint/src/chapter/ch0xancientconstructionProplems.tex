\chapter{Ancient Construction Proplems}
In this chapter we will use the results to solve some of the Ancient construction proplems.

\section{Doubling the cube}
Doubling the cube, also known as the Delian problem, is an ancient geometric problem.
Given the edge of a cube, the problem requires the construction of the edge of a second cube whose volume is double that of the first,
using only a ruler and compass.

We can simplfy the problem as follows:
Let $\mathcal{M} = \{a, b\}$. Let $r := \|a - b\|$ be the distance between $a$ and $b$. Then a Qube with edge $r$ has volume $r^3$.
There is a cube with volume $2r^3$ if and only if $\sqrt[3]{2} \in \mathcal{M}_{\infty}$.
%\begin{theorem}%TODO MAke to problem
%    Let $\mathcal{M} = \{0,1\}$.  
%    \begin{equation*}\sqrt[3]{2} \notin \mathcal{M}_{\infty}\end{equation*}.
%\end{theorem}


\begin{theorem}
    $P := X^3 - 2$ is irreducible over $\mathbb{Q}$.
\label{thm:irreducible_over_Q}
\end{theorem}
\begin{proof}
    Since $\mathbb{Q}$ is $a$ subfield of $\mathbb{C} [X]$, we know that
    \begin{equation*}
        X^3 - 2 = (X - \sqrt[3]{2})(X -\zeta_3 \sqrt[3]{2})(X -\zeta_3^2 \sqrt[3]{2})
    \end{equation*}
    Suppose $P$ is Rational, then
    \begin{equation*}
        X^3 - 2 = (X - a)(X^2 + bX + c)\text{, with } a, b, c \in \mathbb{Q}
    \end{equation*}
    In particular it has a zero in $\mathbb{Q}$, so there is a rational number $a$ such that $a^3 = 2$.\newline
    But we know that $\zeta_3 \sqrt[3]{2}$ and $\zeta_3^2 \sqrt[3]{2}$ are not real numberns and $\sqrt[3]{2}$ is not rational.
    So $P$ is irreducible over $\mathbb{Q}$.
\end{proof}
\begin{theorem}
    The cube can't be doubled using a compass and straightedge.
\end{theorem}
\begin{proof}
    We know that $K_0 = \mathbb{Q}$ and the problem is equivalent to $\sqrt[3]{2} \in \mathcal{M}_{\infty}$.\newline
    \ref{thm:degree_of_constructable_points} tells us that it is sufficient to show that $[\mathbb{Q}(\sqrt[3]{2}):\mathbb{Q}] \ne 2^m$ for some $0 \le m $.
    \newline
    We know that $P := X^3 - 2$ is irreducible over $\mathbb{Q}$ \ref{thm:irreducible_over_Q} and $P(\sqrt[3]{2}) = 0$, therefore $P = \mu_{\sqrt[3]{2},\mathbb{Q}}$.
    So with \ref{thm:degree_of_simple_field_extension} we know $[\mathbb{Q}(\sqrt[3]{2}):\mathbb{Q}] = 3$ and $[\mathbb{Q}(\sqrt[3]{2}):\mathbb{Q}] \ne 2^m$ for some $0 \le m $.
\end{proof}

\section{Trisection of an angle}
Angle trisection is the construction, using only a ruler and compass, of an angle that is one third of a given arbitrary angle.
We can simplfy the problem as follows:
Let $\mathcal{M} = \{a, b, c\}$ with $a, b, c$ not on a line. Let $\alpha := \angle (b - a, c - a) $.
Then $\alpha$ can be trisected if and only if, there is a point $d\in \mathcal{M}_{\infty}$ so that $\angle (b - a, d - a) = \alpha/3$. Using a
"standard" $\mathcal{M} = \{0,1,\exp(\textbf{i} \alpha)\}$ gives us the following problem.
%\begin{theorem}%TODO MAke to problem
%    Let $\mathcal{M} = \{0,1,\exp(\textbf{i} \alpha)\}$. Is $\exp(\textbf{i} \alpha/3) \in \mathcal{M}_{\infty}$?
%\end{theorem}

Let $\mathcal{M} = \{0,1,\exp(\textbf{i} \alpha)\}$ with $\alpha \in (0,2\pi)$. Therfore we know that
\begin{equation*}
    K_0 = \mathbb{Q}(\mathcal{M}\cup \overline{\mathcal{M}}) = \mathbb{Q}(\exp(\textbf{i} \alpha))
\end{equation*}
We need to examine if $\exp(\textbf{i} \alpha/3) \in \mathcal{M}_{\infty}$. Since \ref{thm:degree_of_constructable_points}
that for an postive answer it is nessary that $[\mathbb{Q}(\exp(\textbf{i} \alpha/3)):\mathbb{Q}] = 2^m$ for some $0 \le m $. \newline
Since $\exp(\textbf{i} \alpha/3)$ is zero of $X^3 - \exp(\textbf{i} \alpha)$, we know that $[\mathbb{Q}(\exp(\textbf{i} \alpha/3)):\mathbb{Q}] \le 3$.
Therfore it is equivalent
\begin{enumerate}
    \item $\exp(\textbf{i} \alpha/3) \notin \mathcal{M}_{\infty}$
    \item $\text{degree}(\mu_{\exp(\textbf{i} \alpha/3),\mathbb{Q}}) = 3$
    \item $X^3 - \exp(\textbf{i} \alpha/3)$ is irreducible over $\mathbb{Q}$
\end{enumerate}

\begin{theorem}
    The angle $\pi / 3 = 60^{\circ}$ can't be trisected using a compass and straightedge.
\end{theorem}
\begin{proof}
    We know
    \begin{equation*}
        \exp(\textbf{i}x) = \cos(x) + \textbf{i} \sin(x) \quad \forall x \in \mathbb{R}
    \end{equation*}
    For $\alpha = \pi / 3$ we get
    \begin{equation*}
        \cos(\alpha) = \frac{1}{2}\qquad \text{and}\qquad \sin(\alpha) = \frac{\sqrt{3}}{2}
    \end{equation*}
    Since we know that $\sqrt{r} \in \mathcal{M}_{\infty}$ for $r \in \mathcal{M}_n$
    we see that $\exp(\textbf{i} \alpha) \in \mathcal{M}_{\infty}$ for $\mathcal{M} = \{0,1\}$. \newline
    So we will work with $K_0 = \mathbb{Q}$. \newline
    We also now that if $x\in \mathcal{M}_{\infty}$, then $x.real, x.imag \in \mathcal{M}_{\infty}$. So we focus on $\cos(\alpha/3)$, witch is zero of
    \begin{equation*}
        f := 8 X^3 - 6 X - 1 \in \mathbb{Q}[X]
    \end{equation*}
    Suppose $f$ is reducible over $\mathbb{Q}$, then $f$ has a rational zero $a$, since $f$ is of degree $3$. According to the rational root theorem, a root rational root of $f$ is of the form $\pm \frac{p}{q}$ with $p$ a factor of the constant term and $q$ a factor of the leading coefficient. So the only possible rational zeros of $f$ are
     \begin{equation*}
        \{ \pm 1, \pm \frac{1}{2}, \pm \frac{1}{4}, \pm \frac{1}{8} \}.
     \end{equation*}
     One can check that none of these numbers is a zero of $f$.
     So $f$ is irreducible over $\mathbb{Q}$ and $\cos(\alpha/3) \notin \mathcal{M}_{\infty}$.
     Therefore
        \begin{equation*}
            \exp(\textbf{i} \alpha/3) \notin \mathcal{M}_{\infty}
        \end{equation*}
    So the angle $\pi / 3 = 60^{\circ}$ can't be trisected using a compass and straightedge.
\end{proof}
