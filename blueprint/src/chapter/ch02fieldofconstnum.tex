\chapter{Field of constructable Numbers}
\section[Filed of Constructable Numbers]{Field $M_{\infty}$}
%2.3
\begin{theorem}
    \label{thm:MField}
    \leanok
    \lean{MField}
    For $M\subseteq \mathbb{C}$ with $0,1 \in M$ is $M_{\infty}$ a subfield of $\mathbb{C}$.
\end{theorem}
\begin{proof}
    %TODO: uses M_inf M_M_inf add_M_Inf z_neg_M_inf mul_M_inf inv_M_inf
    %TODO: change proof from lean to latex 
    Chose $z_1, z_2 \in M_{\infty}$,after construction \ref{lem:construction_add}
    we have $z_1 + z_2 \in M_{\infty}$ and construction \ref{lem:construction_neg}
    gives us that for every $z \in M_{\infty}$ we have $-z \in M_{\infty}$. Therefore $M_{\infty}$ is closed under addition, and contains the additive invers.\\
    For multiplication let $z_1 = r_1 \exp(\imath \alpha_1)$ and $z_2 = r_2 \exp(\imath \alpha_2)$ be in $M_{\infty}$. Then we have 
    $$z_1 z_2 = r_1 r_2 \exp(\imath (\alpha_1 + \alpha_2)) \in M_{\infty}$$. By combining the lemmas \ref{construction_add_angle}, \ref{construction_mult_pos_real} and \ref{construction_polar}
    we obtain that $z_1 z_2 \in M_{\infty}$.\\
    For the inverse of $0 \ne z = r \exp(\imath \alpha) \in M_{\infty}$ we know that $$z^{-1} = r^{-1} \exp(\imath(2\pi - \alpha)).$$ We know that $r, \exp(\imath \alpha) \in M_{\infty}$, see construction \ref{construction_polar}.
    Now we can use construction \ref{construction_inv_pos_real} and \ref{lem:construction_conj} to show,
    that $r^{-1}, \exp(\imath(2\pi - \alpha)) = \overline{\exp(\imath\alpha)} \in M_{\infty}$. By applying that $M_{\infty}$ is closed under multiplication, shown above, we get that $z^{-1} \in M_{\infty}$.\\
    Therefore $M_{\infty}$ is a subfield of $\mathbb{C}$.
\end{proof}

\begin{corollary}
    \label{cor:MField_field}
    \leanok
    \lean{MField_field}
    For $M\subseteq \mathbb{C}$ with $0,1 \in M$ is $M_{\infty}$ a field.
\end{corollary}
\begin{proof}
    \uses{thm:MField}
    This is a direct consequence of Theorem \ref{thm:MField} and the fact that a subfield of a field is a field.
\end{proof}


%2.4
\begin{lemma}
    \label{lem:M_inf_properties}
    \leanok
    \lean{MField_i, MField_re_im, MField_re_im', MField_polar}
    \uses{thm:MField}
    For $M\subseteq \mathbb{C}$ with $0,1 \in M$ applies:
    \begin{itemize}
        \item[(i)] $\imath \in M_{\infty}$.
        \item[(ii)] For $z = x + \imath y \in \mathbb{C}$ is equivalent:
            \begin{itemize}
                \item $z \in M_{\infty}$.
                \item $x, y \in M_{\infty}$.
                \item $x, \imath y \in M_{\infty}$.
            \end{itemize}
        \item[(iii)] 
            For $0 \ne z = r \exp(\imath \alpha) \in \mathbb{C}$ is equivalent:
            \begin{itemize}
                \item $z \in M_{\infty}$.
                \item $r,\exp(\imath \alpha) \in M_{\infty}$.
            \end{itemize}
    \end{itemize}
\end{lemma}
\begin{proof}
    %TODO: \uses{} imath_M_inf z_iff_re_im_M_inf; (i_z_imp_z_in_M_inf/ MField_i_mul)
    This lemma is a dierct consequence of section \ref{basic_constructions}.
    \begin{itemize}
        \item[(i):] We kan apply construction \ref{construction_imath}
        \item[(ii):] We can apply construction \ref{construction_re_im} and \ref{construction_imath}.
        \item[(iii):]  We can apply construction \ref{construction_polar}.
    \end{itemize}
\end{proof}

\begin{definition}[quadritc closed field]
    \label{def:quadritc_closed_field}
    \leanok
    \lean{QuadraticClosed}
    A field $K$ is called \emph{quadritc closed} if for all $z \in K$ we have $\sqrt{z} \in K$.
\end{definition}
\begin{remark}
    An equivalent definition is that $K$ is quadritc closed if $$K=\{a^2\mid a \in K\}.$$ 
\end{remark}

%2.6
\begin{lemma}
    \label{lem:M_inf_quad_closed}
    %\leanok
    %\lean{MField_quad_closed}
    \uses{def:set_of_constructable_points, def:quadritc_closed_field}
    For $M\subseteq \mathbb{C}$ with $0,1 \in M$ is $M_{\infty}$ quadritc closed.
\end{lemma}
\begin{proof}
    \uses{cor:MField_field, cor:root_M_inf}
    Combian theorem \ref{cor:MField_field} to get that $M_{\infty}$ is a field and Lemma \ref{cor:root_M_inf} to get that $M_{\infty}$ is a quadritc closed field.
\end{proof}

\begin{definition}
    \label{def:conjugate_set}
    \leanok
    \lean{conj_set}
    For a Set $U \subset \mathbb{C}$ we define the \emph{conjugate set} of $U$ as 
    \begin{equation*}
        Conj(U) = \{z\in \mathbb{C} \mid \exists w\in U: z = \overline{w}\}
    \end{equation*}
\end{definition}

\begin{definition}
    \label{def:conj_closed}
    \leanok
    \lean{ConjClosed}
    We call a subset of $\C$ \emph{conjugate closed} if $M= Conj(M)$.
\end{definition}

\begin{lemma}
    \label{lem:conj_MField}
    \leanok
    \lean{MField_conj}
    $M_{\infty}$ is conjugate closed.
\end{lemma}
\begin{proof}
    We can apply construction \ref{lem:construction_conj} and the fact that $\overline{\overline z} = z$ for all $z \in\mathbb{C}$.
\end{proof}


\section{Joshuas Kapitel}

\begin{definition}
    The degree of $x$ over $K$ is
    \begin{equation*}
        [x:K] :=\text{degree}(\mu_{x,K})
        \end{equation*}
        with $\mu_{x,K}$ the minimal polynomial of $x$ over $K$. \newline
    The degree of $L/K$ is the dimension of $L$ as a $K$-vector space and is denoted by
    \begin{equation*}
        [L:K].
    \end{equation*}
\end{definition}

\begin{theorem}
\label{thm:degree_of_simple_field_extension}
    Let $L/K$ be a simple field extension with $L = K(x)$. Then
    \begin{equation*}
        [L:K] = [x:K].
    \end{equation*}
\end{theorem}
\begin{proof}
    In Mathlib: theorem IntermediateField.adjoin.finrank
\end{proof}

\begin{definition}
    \label{def:K_M_0}
    Let $\mathcal(M)\subseteq\mathbb{C}$ with $0,1 \in \mathcal{M}$
    \begin{equation*}
        K_0 := \mathbb{Q}(\mathcal{M}\cup \overline{\mathcal{M}})
    \end{equation*}
    with $\overline{\mathcal{M}} := \{ \overline{z} = x - \textbf{i}y \mid z = x+\textbf{i}y  \in \mathcal{M} \}$.
\end{definition}

TODO: Add $K_i$ and $K_{\infty}$ and a Lemma that states that $K_{\infty} = M_{\infty}$. But this is not needed for the project.

%3.20
\begin{lemma}
    \label{lem:conj_of_adjoin}
    Let $K$ be a subfield of $\mathbb{C}$ with $conj(K)=K$, and $ M \subset \mathbb{C}$ be a subset with $M = conj(M)$. Then holds
    \begin{equation*}
        conj(K(M)) = K(M).
    \end{equation*}
\end{lemma}
\begin{proof}
    We know that the complex conjugation is a field automorphism $\overline{\cdot }: \mathbb{C} \to \mathbb{C}$, whit $\overline{\overline{z}} = z$ for all $z \in \mathbb{C}$. Therefore we have
    $conj(K(M))$ is a subfield of $\mathbb{C}$.\\
    Since $K$ and $M$ are subsets of $K(m)$ we have $conj(K) = K$ and $conj(M) = M$ are subsets of $conj(K(M))$. Therefore $$K(M) \subset conj(K(M)).$$
    If we know apply $conj$ to both sides we get
    $$conj(K(M)) \subset conj(conj(K(M))) = K(M).$$
    Therefore $conj(K(M)) = K(M)$.
\end{proof}

%3.19
\begin{lemma}
    \label{lem:dergee2_iff_adjoint_sqrt}
    Let $K\subset L \subset \mathbb{C}$ be fields. Then the following is equivalent:
    \begin{itemize}
        \item $[L:K] = 2$.
        \item There is a $w \in K$ with $\sqrt{w} \notin K$ and $L = K(\sqrt{w})$.
    \end{itemize}
\end{lemma}
\begin{proof}
    $(ii)\implies (i)$: Let $w$ be as in $(ii)$.Then $\sqrt{w}$ is a root of $X^2 - w \in K[X]$. Since $\sqrt{w} \notin K$ this polynomial is irreducible in $K[X]$. Therefore $[L:K] = 2$.\\
    $(i)\implies (ii)$: Let $[L:K] = 2$ and $a \in L \setminus K$. Then $K(\alpha) = L$ and 
    $$\mu_{\alpha, K}=X^2 + bX + c \quad b,c \in K$$
    This implies 
    $$\alpha = -\frac{b}{2} \pm \sqrt{\frac{b^2}{4} - c} $$
    Now let $w := \frac{b^2}{4} - c \in K$ then we get $L = K(\alpha) = K(\sqrt{w})$.
\end{proof}


%TODO: add lemma wich $K_0 \subset M_{\infty}$

%3.21
\begin{theorem}[constructable iff chain dergee2]
    \label{thm:Z_in_Minf_1}
    For $z \in \mathbb{C}$ is equivalent:
    \begin{itemize}
        \item $z \in M_{\infty}$
        \item There is an intermidiate field $L$ of $\mathbb{C}/K_0$ whit $z \in L$, so that $L$ aries from $K_0$ by a sequence of adjunctions of square roots.
        \item There is an $n\le 0$ and chain 
            $$K_0 = L_1 \subset L_2 \subset \ldots \subset L_n \subset \mathbb{C}$$
            of subfields of $\mathbb{C}$ so that $z \in L_n$ and 
            $$ [L_{i+1}:L_i] = 2 \quad \text{for} \quad i = 0, \ldots, n-1.$$
    \end{itemize}
\end{theorem}

\begin{remark}
    \label{rem:Z_in_Minf_1}
    In this cases the $[K_0(z):K_0] = 2^m$ for some $0 \le m \le n$.
\end{remark}
\begin{proof}[Proof of Theorem \ref{thm:Z_in_Minf_1}]
    \uses{ lem:dergee2_iff_adjoint_sqrt, Intersection_circle_circle, Intersection_line_circle, Intersection_line_line, lem:conj_of_adjoin, lem:M_inf_quad_closed}
    $(ii) \iff (iii)$ Follows from the definition and Lemma \ref{lem:dergee2_iff_adjoint_sqrt}.\\
    $(i) \implies (ii)$: We assume that $z$ is derived from $K_0$ by (ZL1), (ZL2) and (ZL3). %TODO: add ref
    Lemma \ref{Intersection_circle_circle}, \ref{Intersection_line_circle} and \ref{Intersection_line_line} in combination with Lemma \ref{lem:conj_of_adjoin} 
    imply that ther is an $w \in K_0$ so that $z \in K_0(\sqrt{w})$. Since $\overline{w} \in K_0$, by definition. We define
    $$K_1 := K_0(\sqrt{w},\sqrt{\overline{w}}).$$
    $K_1$ derives from $K_0$ sucsessively by adjunction of square roots. Note that $K_1 = conj(K_1)$, by Lemma \ref{lem:conj_of_adjoin} and the fact that $\sqrt{\overline{w}} = \overline{\sqrt{w}}$.
    Because of the definition of $$M_{\infty} = \bigcup_{n=0}^{\infty} M_n,$$
    he statement follows by induction.\\ %TODO: Elaborate on induction for lean
    $(ii) \implies (i)$: We know that $M_{\infty}$ is an quadritc complet field \ref{lem:M_inf_quad_closed}, i.e. $\sqrt{z} \in M_{\infty}$ for all $z \in M_{\infty}$.
    Futhermore we know that $K_0\subseteq M_{\infty} \subseteq \mathbb{C}$, %TODO: add ref
    therefore every field $L\subseteq \mathbb{C}$ which is derived from $K_0$ by a sequence of adjunctions of square roots is contaianted in $M_{\infty}$.
\end{proof}