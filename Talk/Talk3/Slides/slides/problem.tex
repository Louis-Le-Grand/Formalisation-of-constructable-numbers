\begin{frame}[fragile]
\begin{columns}
    \begin{column}{0.5\textwidth}
\begin{figure}
    

\begin{tikzpicture}
    \draw[dotted, very thick,medgrey] (1,-1,-1)--(-1,-1,-1)--(-1,1,-1);
    \draw[dotted, very thick,medgrey] (-1,-1,-1)--(-1,-1,1);


    \draw[thick,opacity=0.2,fill=Ccolor]  (1,1,1)--(1,-1,1)--(-1,-1,1)--(-1,1,1)--cycle; % top
    \draw[thick,opacity=0.2,fill=Ccolor] (1,1,1)--(1,1,-1)--(-1,1,-1)--(-1,1,1)--cycle; % front
    \draw[thick,opacity=0.2,fill=Ccolor] (1,1,-1)--(1,1,1)--(1,-1,1)--(1,-1,-1)--cycle;
    \draw[thick, style={ |-|}] (1.2,-1,-1) -- (1.2,0,-1)  node[right] {$a$} -- (1.2,1,-1);
\end{tikzpicture}

\end{figure}
This cube has volume $a^3$
\end{column}

\pause

\begin{column}{0.5\textwidth}
    If we have cube width volume $2a^3$
    \newline
    \newline
    \scalebox{1.2}{\begin{minipage}{0.80\textwidth}
    \begin{figure}
    \centering
    \begin{tikzpicture}
        \draw[dotted, very thick,medgrey] (1,-1,-1)--(-1,-1,-1)--(-1,1,-1);
        \draw[dotted, very thick,medgrey] (-1,-1,-1)--(-1,-1,1);


        \draw[thick,opacity=0.2,fill=Ccolor]  (1,1,1)--(1,-1,1)--(-1,-1,1)--(-1,1,1)--cycle; % top
        \draw[thick,opacity=0.2,fill=Ccolor] (1,1,1)--(1,1,-1)--(-1,1,-1)--(-1,1,1)--cycle; % front
        \draw[thick,opacity=0.2,fill=Ccolor] (1,1,-1)--(1,1,1)--(1,-1,1)--(1,-1,-1)--cycle;
        \pause
        \draw[thick, style={ |-|}] (1.2,-1,-1) -- (1.2,0,-1)  node[right] {$\sqrt[3]{2}a$} -- (1.2,1,-1);
    \end{tikzpicture}
    %\label{Cube with Volume $2a^3$}
    \end{figure}
    \end{minipage}}
\end{column}
\end{columns}
\end{frame}

% \begin{frame}
%     \begin{tikzpicture}
%         \pgfmathsetmacro{\cubex}{3}
%         \pgfmathsetmacro{\cubey}{3}
%         \pgfmathsetmacro{\cubez}{3}
%         \draw[thick, opacity=0.2, fill=grey] (0,0,0) -- ++(-\cubex,0,0) -- ++(0,-\cubey,0) -- ++(\cubex,0,0) -- cycle;
%         \draw[thick, opacity=0.2, fill=grey] (0,0,0) -- ++(0,0,-\cubez) -- ++(0,-\cubey,0) -- ++(0,0,\cubez) -- cycle;
%         \draw[thick, opacity=0.2, fill=grey] (0,0,0) -- ++(-\cubex,0,0) -- ++(0,0,-\cubez) -- ++(\cubex,0,0) -- cycle;
%     \end{tikzpicture}
% \end{frame}

\begin{frame}
    \begin{problem}
        Let $\mathcal{M} = \{0,1\}$.  
        $$\text{Is }\sqrt[3]{2} \overset{?}{\in} \mathcal{M}_{\infty}?$$
    \end{problem}
\end{frame}

\begin{frame}
    \begin{lemma}
        $P := X^3 - 2$ is irreducible over $\mathbb{Q}$.
    \label{lem:irreducible_over_Q}
    \leanok
    \lean{P_irreducible}
    \end{lemma}
    \begin{proof}
        \leanok
        \uses{lem:irrational_thirdroot_two}
        Since $\mathbb{Q}$ is $a$ subfield of $\mathbb{C} [X]$, we know that
        \begin{equation*}
            X^3 - 2 = (X - \sqrt[3]{2})(X -\zeta_3 \sqrt[3]{2})(X -\zeta_3^2 \sqrt[3]{2})
        \end{equation*}
        Suppose $P$ is reducible, then
        \begin{equation*}
            X^3 - 2 = (X - a)(X^2 + bX + c)\text{, with } a, b, c \in \mathbb{Q}
        \end{equation*}
        In particular it has a zero in $\mathbb{Q}$, so there is a rational number $a$ such that $a^3 = 2$.\newline
        But we know that $\zeta_3 \sqrt[3]{2}$ and $\zeta_3^2 \sqrt[3]{2}$ are not real numbers and $\sqrt[3]{2}$ is not rational.
        So $P$ is irreducible over $\mathbb{Q}$.
    \end{proof}
\end{frame}

\begin{frame}
    \begin{theorem}
        \label{thm:third_root_of_two_not_in_M_inf}
        \leanok
        \lean{third_root_of_two_not_in_M_inf}
        The cube can't be doubled using a compass and straightedge.
    \end{theorem}
    \begin{proof}
        \leanok
        \uses{lem:irrational_thirdroot_two, lem:irreducible_over_Q, cor:Classfication_z_in_M_inf_2m_not}
        By applying the corollary classification, it is sufficient to proof that no $m\in \N$ exists such that
        $$2^m\overset{?}{=}[\sqrt[3]{2}:\mathcal{K}_0]\overset{0,1\in\Q}{=} [\sqrt[3]{2}:\Q] = \text{degree}(\mu_{\Q,\sqrt[3]{2}}).$$
        Since $P$ is irreducible over $\mathbb{Q}$, monic and has $\sqrt[3]{2}$ as a zero, we know that $[\mathbb{Q}(\sqrt[3]{2}):\mathbb{Q}] = 3$.
        And since $3 \ne 1$ and  $$ 3  \equiv_2 1 \neq 0  \equiv_2 = 2^m \qquad\forall m \in \N_{\ge 1}$$
        we can conclude that the cube can't be doubled using a compass and straightedge.
        
    \end{proof}
\end{frame}
