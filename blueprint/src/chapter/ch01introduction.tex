\chapter{Introduction}

\section{Defining circles, lines and the set of constructable points}
First we need to define what construction using a ruler and compass means.
We will use $\mathbb{C}$ as plane of drawing and $\mathcal{M} \subset \mathbb{C}$ as the set of constructed points.

\begin{definition}[Line]
    \label{def:line}
    \lean{line}
    \leanok
    A line $l$ through two points $x,y\in\mathbb{C}$ for $x\ne y$ is definded by the set: $$l:=\{\lambda x+(1-\lambda)y\mid\lambda\in\mathbb{R}\}.$$
\end{definition}

\begin{definition}[Circle]
    \label{def:circle}
    \lean{circle}
    \leanok
    A circle $c$ with center $z\in\mathbb{C}$ and radius $r\in\mathbb{R}_{\ge 0}$ is defined by the set: $$c:=\{z\in\mathbb{C} \mid\|z-c\|=r\}.$$
\end{definition}

\begin{definition}[Set of lines]
    \label{def:set_of_lines}
    \lean{L}
    \leanok
    \uses{def:line}
    $\mathcal{L(M)}$ is the set of all real straight lines $l$, with $| l\cap \mathcal{M} |\ge 2$. As set this would be
    \begin{equation*}
        \mathcal{L(M)} := \{l \mid l = \{x,y\} \text{ with }x,y \in \mathcal{M} \land x \neq y\}.
    \end{equation*}
\end{definition}

\begin{definition}[Set of circles]
    \label{def:set_of_circles}
    \lean{C}
    \leanok
    \uses{def:circle}
    $\mathcal{C(M)}$ is the set of all circles in $\mathbb{C}$, with center in $\mathcal{M}$ and radius of $\mathcal{C}$ is the distence of two points in $\mathcal{M}$. As equation this would be:
    $$\mathcal{C(M)}:=\{c\mid c = \langle c, \dist r_1 r_2 \rangle \text{ with } c,r_1,r_2\in M\}.$$
\end{definition}

\begin{definition}[Ruels to constructed a Point]
    \label{def:rules_to_constructed_a_point}
    \lean{ill,ilc,icc,ICL_M}
    \leanok
    \uses{def:set_of_lines, def:set_of_circles}
    We define operation that can be used to constructed new Points.
    \begin{enumerate}
        \item $(ILL)$ is the cut of two lines in $\mathcal{L(M)}$.
        \item $(ILC)$ is the cut of a line in $\mathcal{L(M)}$ and a circle in $\mathcal{C(M)}$.
        \item $(ICC)$ is the cut of two circles in $\mathcal{C(M)}$.
    \end{enumerate}
    $ICL(\mathcal{M})$ is the set $\mathcal{M}$ combeined with of all points that can be constructed using the operations $(ILL)$, $(ILC)$ and $(ICC)$.
\end{definition}

\begin{definition}[Set of Constructable Points]
    \label{def:set_of_constructable_points}
    \lean{M_I,M_inf}
    \leanok
    \uses{def:rules_to_constructed_a_point}
    We define inductively the the chain
    \begin{equation*}
        \mathcal{M}_0 \subseteq \mathcal{M}_1 \subseteq \mathcal{M}_2 \subseteq \dots
    \end{equation*}
    with $\mathcal{M}_0 = \mathcal{M}$ and $\mathcal{M}_{n+1} = ZL(\mathcal{M}_n)$.\newline
    And call $\mathcal{M}_{\infty} = \bigcup_{n \in \mathbb{N}} \mathcal{M}_n$ the set of all constructable points.
\end{definition}

\section[The set of constructable points]{The Set $M_{\infty}$}\label[section]{set_of_constructable_points}

\section{Basic constructions}\label{basic_constructions}
For the following constructions we assume that $M\subseteq \mathbb{C}$ with $0,1 \in M$.

%2.3.1
\begin{lemma}[Negative complex numbers]
    \label{lem:construction_neg}
    For $z \in M_{\infty}$ is $-z \in M_{\infty}$.
\end{lemma}

\begin{proof}
    %TODO: add proof, and Construction Idea
\end{proof}

%2.3.2
\begin{lemma}[Addition of complex numbers]
    \label{lem:construction_add}
    For $z_1, z_2 \in M_{\infty}$ is $z_1 + z_2 \in M_{\infty}$.
\end{lemma}

\begin{proof}
    %TODO: add proof, and Construction Idea
\end{proof}

%2.3.3
%TODO: Comper to mirrowed version
\begin{lemma}[Complex conjugation]
    \label{lem:construction_conj}
    For $z \in M_{\infty}$ is $\overline{z} \in M_{\infty}$.
\end{lemma}

\begin{proof}
    %TODO: add proof, and Construction Idea
\end{proof}

%2.3.4
\begin{lemma}[Mitpiont of two complex numbers]
    \label{lem:construction_midpoint}
    For $z_1, z_2 \in M_{\infty}$ is $\frac{z_1 + z_2}{2} \in M_{\infty}$.
\end{lemma}
%TODO: add proof, and Construction Idea

%2.3.5
\begin{lemma}[Halving of Angle]
    \label{lem:construction_halving_angle}
    For $z = r \exp(\imath \alpha) \in M_{\infty}$ is $\exp(\imath \alpha / 2) \in M_{\infty}$.
\end{lemma}

%TODO: add proof, and Construction Idea

%2.3.6
%TODO: maybe write as \exp(\imath \alpha) + \exp(\imath \beta) = \exp(\imath \gamma) \in M_{\infty}
\begin{lemma}[Addition of Angles]
    \label{lem:construction_add_angle}
    For two given angle $\alpha$ (given by two streches $[a,b]$ and $[c,d]$) and $\beta$ (given by two streches $[e,f]$ and $[g,h]$), with $ a,b,c,d,e,f,g,h \in M_{\infty}$, the addition of the angles is constructable.
\end{lemma}
%TODO: add proof, and Construction Idea, 

%2.3.7
\begin{lemma}[multiplication of positive real numbers]
    \label{lem:construction_mult_pos_real}
    For $r_1, r_2 > 0 \in \mathbb{R}\cup M_{\infty}$ is $r_1 r_2 \in M_{\infty}$.
\end{lemma}
%TODO: add proof, and Construction Idea

%2.3.8
\begin{lemma}[Inverse of positive real numbers]
    \label{lem:construction_inv_pos_real}
    For $r > 0\in \mathbb{R}\cup M_{\infty}$ is $r^{-1} \in M_{\infty}$.
\end{lemma}
%TODO: add proof, and Construction Idea

%2.3.9
\begin{lemma}[Root of positive real numbers]
    \label{lem:construction_sqrt_pos_real}
    For $r > 0 \in \mathbb{R}\cup M_{\infty}$ is $\sqrt{r} \in M_{\infty}$.
\end{lemma}
%TODO: add proof, and Construction Idea

%2.3.10
\begin{lemma}[Polarkoordinates of complex numbers]
    \label{lem:construction_polar}
    For $z = r \exp(\imath \alpha) \in M_{\infty}$ is $r, \allowbreak \exp(\imath \alpha) \in M_{\infty}$.
\end{lemma}
%TODO: add proof, and Construction Idea

%2.3.11
\begin{lemma}[Real and Imaginary part of complex numbers]
    \label{lem:construction_re_im}
    For $0 \ne z = x + \imath y \in M_{\infty}$ is $x, \imath y \in M_{\infty}$.
\end{lemma}
%TODO: add proof, and Construction Idea

%2.3.12
\begin{lemma}[Construction of $\imath$]
    \label{lem:construction_imath}
    $\imath \in M_{\infty}$.
\end{lemma}
%TODO: add proof, and Construction Idea