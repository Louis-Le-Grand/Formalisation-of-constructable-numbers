\chapter{Introduction}

\section{Defining circles and lines and the set of constructable points}
First we need to define what construction using a ruler and compass means.
We will use $\mathbb{C}$ as plane of drawing and $\mathcal{M} \subset \mathbb{C}$ as the set of constructed points.


\begin{definition}
    $\mathcal{G(M)}$ is the set of all real straight lines $\mathcal{G}$, with $| \mathcal{G} \cap \mathcal{M} |\ge 2$.\newline
    $\mathcal{C(M)}$ is the set of all circles in $\mathbb{C}$, with center in $\mathcal{M}$ and radius of $\mathcal{C}$ is the distence of two points in $\mathcal{M}$.
\end{definition}

\begin{definition}
    We define operation that can be used to constructed new Points.
    \begin{enumerate}
        \item $(ZL 1)$ is the cut of two lines in $\mathcal{G(M)}$.
        \item $(ZL 2)$ is the cut of a line in $\mathcal{G(M)}$ and a circle in $\mathcal{C(M)}$.
        \item $(ZL 3)$ is the cut of two circles in $\mathcal{C(M)}$.
    \end{enumerate}
    $ZL(\mathcal{M})$ is the set $\mathcal{M}$ combeined with of all points that can be constructed using the operations $(ZL 1)$, $(ZL 2)$ and $(ZL 3)$.
\end{definition}

\begin{definition}
    We define inductively the the chain
    \begin{equation*}
        \mathcal{M}_0 \subseteq \mathcal{M}_1 \subseteq \mathcal{M}_2 \subseteq \dots
    \end{equation*}
    with $\mathcal{M}_0 = \mathcal{M}$ and $\mathcal{M}_{n+1} = ZL(\mathcal{M}_n)$.\newline
    And call $\mathcal{M}_{\infty} = \bigcup_{n \in \mathbb{N}} \mathcal{M}_n$ the set of all constructable points.
\end{definition}

\section{Basic constructions}\label[section]{lem:basic_constructions}
For the following constructions we assume that $M\subseteq \mathbb{C}$ with $0,1 \in M$.

%2.3.1
\begin{lemma}[Negative complex numbers]
    \label{lem:construction_neg}
    For $z \in M_{\infty}$ is $-z \in M_{\infty}$.
\end{lemma}

\begin{proof}
    %TODO: add proof, and Construction Idea
\end{proof}

%2.3.2
\begin{lemma}[Addition of complex numbers]
    \label{lem:construction_add}
    For $z_1, z_2 \in M_{\infty}$ is $z_1 + z_2 \in M_{\infty}$.
\end{lemma}

\begin{proof}
    %TODO: add proof, and Construction Idea
\end{proof}

%2.3.3
%TODO: Comper to mirrowed version
\begin{lemma}[Complex conjugation]
    \label{lem:construction_conj}
    For $z \in M_{\infty}$ is $\overline{z} \in M_{\infty}$.
\end{lemma}

\begin{proof}
    %TODO: add proof, and Construction Idea
\end{proof}

%2.3.4
\begin{lemma}[Mitpiont of two complex numbers]
    \label{lem:construction_midpoint}
    For $z_1, z_2 \in M_{\infty}$ is $\frac{z_1 + z_2}{2} \in M_{\infty}$.
\end{lemma}
%TODO: add proof, and Construction Idea

%2.3.5
\begin{lemma}[Halving of Angle]
    \label{lem:construction_halving_angle}
    For $z = r \exp(\imath \alpha) \in M_{\infty}$ is $\exp(\imath \alpha / 2) \in M_{\infty}$.
\end{lemma}

%TODO: add proof, and Construction Idea

%2.3.6
%TODO: maybe write as \exp(\imath \alpha) + \exp(\imath \beta) = \exp(\imath \gamma) \in M_{\infty}
\begin{lemma}[Addition of Angles]
    \label{lem:construction_add_angle}
    For two given angle $\alpha$ (given by two streches $[a,b]$ and $[c,d]$) and $\beta$ (given by two streches $[e,f]$ and $[g,h]$), with $ a,b,c,d,e,f,g,h \in M_{\infty}$, the addition of the angles is constructable.
\end{lemma}
%TODO: add proof, and Construction Idea, 

%2.3.7
\begin{lemma}[multiplication of positive real numbers]
    \label{lem:construction_mult_pos_real}
    For $r_1, r_2 > 0 \in \mathbb{R}\cup M_{\infty}$ is $r_1 r_2 \in M_{\infty}$.
\end{lemma}
%TODO: add proof, and Construction Idea

%2.3.8
\begin{lemma}[Inverse of positive real numbers]
    \label{lem:construction_inv_pos_real}
    For $r > 0\in \mathbb{R}\cup M_{\infty}$ is $r^{-1} \in M_{\infty}$.
\end{lemma}
%TODO: add proof, and Construction Idea

%2.3.9
\begin{lemma}[Root of positive real numbers]
    \label{lem:construction_sqrt_pos_real}
    For $r > 0 \in \mathbb{R}\cup M_{\infty}$ is $\sqrt{r} \in M_{\infty}$.
\end{lemma}
%TODO: add proof, and Construction Idea

%2.3.10
\begin{lemma}[Polarkoordinates of complex numbers]
    \label{lem:construction_polar}
    For $z = r \exp(\imath \alpha) \in M_{\infty}$ is $r, \allowbreak \exp(\imath \alpha) \in M_{\infty}$.
\end{lemma}
%TODO: add proof, and Construction Idea

%2.3.11
\begin{lemma}[Real and Imaginary part of complex numbers]
    \label{lem:construction_re_im}
    For $0 \ne z = x + \imath y \in M_{\infty}$ is $x, \imath y \in M_{\infty}$.
\end{lemma}
%TODO: add proof, and Construction Idea

%2.3.12
\begin{lemma}[Construction of $\imath$]
    \label{lem:construction_imath}
    $\imath \in M_{\infty}$.
\end{lemma}
%TODO: add proof, and Construction Idea