\chapter{Appendix: A collection of results which are needed in the proof.}
\label{ch_bestiary}
In this (temporary, unorganised) appendix I list a whole host of definitions and theorems which were known to humanity by the end of the univers \cite{JAN_SCHRÖER:2023}. These definitions and theorems will find their way into more relevant sections of the blueprint once I have time. Note also that many of the \emph{definitions} here are yet to be formalised in Lean, and this needs to be done before we can start talking about formalising theorems.
%We need the a debug placeholder for BibTex Stuff:


\section*{Stuff I'm working on}




%3.14
\begin{lemma}
    \label{lem:re_im_in_L}
    Let $L$ be a subfield of $\mathbb{C}$, with $L = conj(L)$. For all $z = x + \imath y \in L$ we have $x, \imath y \in L$.
\end{lemma}
\begin{proof}
    Let $z = x + \imath y \in L$. According to prerequisite we have $\overline{Z}=x-\imath y \in L$. This implies
    \begin{equation*}
        \frac{z + \overline{z}}{2} = x \in L
    \end{equation*}
    and therfore also $\imath y = z - x \in L$.
\end{proof}

%3.15
\begin{lemma}
    \label{lem:dist_sqard_in_L}
    Let $L$ be a subfield of $\mathbb{C}$, with $L = conj(L)$, and $z_1, z_2 \in L$.
    For $r := \|z_1-z_2\|$ we get that $r^2 \in L$.
\end{lemma}
\begin{proof}
    \uses{lem:re_im_in_L}
    For $z_1 = x_1 + \imath y_1$ and $z_2 = x_2 + \imath y_2$ we have
    \begin{equation*}
        r = \|z_1 - z_2\| = \sqrt{(x_1 - x_2)^2 + (y_1 - y_2)^2}
    \end{equation*}
    and therefore
    \begin{equation*}
        r^2 = (x_1 - x_2)^2 + (y_1 - y_2)^2 
    \end{equation*}
    After applying Lemma \ref{lem:re_im_in_L} we get $r^2 \in L$.
\end{proof}

%3.16
\begin{lemma}
    \label{Lem:Intersection_line_line}
    Let $L$ be a subfield of $\mathbb{C}$, with $L = conj(L)$. For $i = 1,2,3,4$ let $z_i = x_i + \imath y_i \in L$ with $z_1 \ne z_2$ and $z_3 \ne z_4$. Define
    %TODO: write in two lines
    \begin{equation*}\begin{aligned}
        G_1 := \{\lambda z_1 + (1-\lambda)z_2 \mid \lambda \in \mathbb{R}\},\\
        G_2 := \{\mu z_3 + (1-\mu)z_4 \mid \mu \in \mathbb{R}\}.
    \end{aligned} \end{equation*}
    If $G_1 \cap G_2 \ne \emptyset$ and $G_1 \ne G_2$. It is equivalent 
    \begin{itemize}
        \item $z\in G_1 \cap G_2$.
        \item Ther are $\lambda, \mu \in \mathbb{R}$ such that 
        $$\lambda(x_1 - x_2)+\mu(x_4-x_3) = x_4-x_2,\\
        \lambda(\imath y_1 - \imath y_2)+\mu(\imath y_4-\imath y_3) = \imath y_4-\imath y_2,$$
        and $z = \lambda z_1 + (1-\lambda)z_2$.
    \end{itemize}
    In this case $z \in L$.
\end{lemma}
\begin{proof}
    \uses{lem:re_im_in_L}
    It is clear that $z\in G_1\cap G_2$ holds if and only if ther are $\lambda, \mu \in \mathbb{R}$ such that
    \begin{equation*}
        \lambda z_1 + (1-\lambda)z_2 = \mu z_3 + (1-\mu)z_4
    \end{equation*}
    We can rewrite this as to get the desired result. %TODO: add Rewrite
    Since $L = conj(L)$ we have $x_i, \imath y_i \in L$ \ref{lem:re_im_in_L}. 
    If ther is a solution i $\mathbb{R}^2$ for $\mu, \lambda$, then ther is a soltion in $L^2$.
    %TODO: add more
    So $\lambda$ and therefore $z$ is in $L$.
\end{proof}

%3.17
\begin{lemma}
    \label{Lem:Intersection_line_circle}
    Let $L$ be a subfield of $\mathbb{C}$, with $L = conj(L)$. For $i = 1,2,3$ let $z_i = x_i + \imath y_i \in L$ with $z_1 \ne z_2$, and let $r > 0$ in $\mathbb{R}$ with $r^2 \in L$. Define
    \begin{equation*}\begin{aligned}
        G := \{\lambda z_1 + (1-\lambda)z_2 \mid \lambda \in \mathbb{R}\},\\
        C := \{z = x + \imath y \in \mathbb{C} \mid \|z - z_3\| = r\}.
    \end{aligned} \end{equation*}
    If $G \cap C \ne \emptyset$. Then the following is equivalent:
    \begin{itemize}
        \item $z\in G \cap C$.
        \item There is a $\lambda \in \mathbb{R}$ with $\lambda^2 a+ \lambda b + c = 0$ where
        \begin{align*}
            a &:= (x_1 - x_2)^2 + (\imath y_1 - \imath y_2)^2,\\
            b &:= 2(x_1 - x_2)(x_2 - x_3) - 2(\imath y_1 - \imath y_2)(\imath y_2 - \imath y_3),\\
            c &:= (x_2 - x_3)^2 + (\imath y_2 - \imath y_3)^2 - r^2,
        \end{align*}
        and $z = \lambda z_1 + (1-\lambda)z_2$.
    \end{itemize}
    In this case $z \in L(\sqrt{w})$ for an $w \in L$.
\end{lemma}

\begin{proof}
%TODO: add proof
\end{proof}

%3.18
\begin{lemma}
    \label{Lem:Intersection_circle_circle}
    Let $L$ be a subfield of $\mathbb{C}$, with $L = conj(L)$. For $i = 1,2 $ let $z_i = x_i + \imath y_i \in L$ with $z_1 \ne z_2$ and let $r_i > 0$ in $\mathbb{R}$ with $r_i^2 \in L$. Define
    \begin{equation*} \begin{aligned}
        C_1 := \{z = x + \imath y \in \mathbb{C} \mid \|z - z_1\| = r_1\},\\
        C_2 := \{z = x + \imath y \in \mathbb{C} \mid \|z - z_2\| = r_2\}.
    \end{aligned} \end{equation*}
    If $C_1 \cap C_2 \ne \emptyset$ and $C_1 \ne C_2$. Then 
    $$G := \{x+\imath y \in \mathbb{C} \mid 2(x_2 - x_1)x - 2(\imath y_2 - \imath y_1)\imath y = r_1^2 - r_2^2 + x_2^2 - x_1^2 + (\imath y_2)^2 - (\imath y_1)^2\} $$
    is a real line, and $$ C_1 \cap C_2 = G \cap C_1 = G \cap C_2. $$
    For $z \in C_1 \cap C_2$ there is a $w \in L$ such that $z \in L(\sqrt{w})$.
\end{lemma}
\begin{proof}
    %TODO: add proof
\end{proof}


\section{Kapitel 2 stuff for safekeeping}
\begin{definition}
    The degree of $x$ over $K$ is
    \begin{equation*}
        [x:K] :=\text{degree}(\mu_{x,K})
        \end{equation*}
        with $\mu_{x,K}$ the minimal polynomial of $x$ over $K$. \newline
    The degree of $L/K$ is the dimension of $L$ as a $K$-vector space and is denoted by
    \begin{equation*}
        [L:K].
    \end{equation*}
\end{definition}

\begin{theorem}
\label{thm:degree_of_simple_field_extension}
    Let $L/K$ be a simple field extension with $L = K(x)$. Then
    \begin{equation*}
        [L:K] = [x:K].
    \end{equation*}
\end{theorem}
\begin{proof}
    In Mathlib: theorem IntermediateField.adjoin.finrank
\end{proof}
\begin{definition}
    Let $\mathcal(M)\subseteq\mathbb{C}$ with $0,1 \mathbb{N} \mathcal{M}$
    \begin{equation*}
        K_0 := \mathbb{Q}(\mathcal{M}\cup \overline{\mathcal{M}})
    \end{equation*}
    with $\overline{\mathcal{M}} := \{ \overline{z} = x - \textbf{i}y \mid z = x+\textbf{i}y  \mathbb{N} \mathcal{M} \}$.
\end{definition}
%\begin{remark}
%    We know that $K_0$ is a field and $\mathcal{M} \subseteq K_0\subseteq \mathcal{M}_{\mathbb{N}fty} \subseteq \mathbb{C}$.
%\end{remark}
\begin{theorem}
\label{thm:degree_of_constructable_points}
    Let $\mathcal{M}\subseteq\mathbb{C}$ with $0,1 \mathbb{N} \mathcal{M}$ and $K_0 := \mathbb{Q}(\mathcal{M}\cup \overline{\mathcal{M}})$.
    Then for $z \mathbb{N} \mathcal{M}_{\mathbb{N}fty}$ is equvalent:
    \begin{enumerate}
        \item $z \mathbb{N} \mathcal{M}_{\mathbb{N}fty}$
        \item There is an $n \mathbb{N} \mathbb{N}$ and a chain \begin{equation*}
            K_0 = L_0 \subset L_1 \subset \dots \subset L_n \subset \mathbb{C}
        \end{equation*}
        of subfields of $\mathbb{C}$ such that $z \mathbb{N} L_n$ and $[L_i:L_{i-1}] =2$ for $1\le i\le n$ %$L_{i+1}$ is obtained from $L_i$ by adjoining a square root.
    \end{enumerate}
    In this case $[K_0(z):K_0] = 2^m$ for some $0 \le m \le n$.
\end{theorem}
\begin{remark}
    Theorem \ref{thm:degree_of_constructable_points} tells us that it is sufficient to show that $[K_0(z):K_0] \ne 2^m$ for some $0 \le m $, witch we will use to show that $\sqrt[3]{2} \notin \mathcal{M}_{\mathbb{N}fty}$ and $\exp(\textbf{i} \alpha/3) \notin \mathcal{M}_{\mathbb{N}fty}$ for some $\alpha$.
\end{remark}
\begin{proof}[Proof of Theorem \ref{thm:degree_of_constructable_points}]
    TODO %TODO
\end{proof}

\section{Safekeeping constructions}

%2.3.4
\begin{lemma}[Mitpiont of two complex numbers]
    \label{lem:construction_midpoint}
    For $z_1, z_2 \in M_{\infty}$ is $\frac{z_1 + z_2}{2} \in M_{\infty}$.
\end{lemma}
%TODO: add proof, and Construction Idea

%2.3.5
\begin{lemma}[Halving of Angle]
    \label{lem:construction_halving_angle}
    For $z = r \exp(\imath \alpha) \in M_{\infty}$ is $\exp(\imath \alpha / 2) \in M_{\infty}$.
\end{lemma}

%TODO: add proof, and Construction Idea

%2.3.6
%TODO: maybe write as \exp(\imath \alpha) + \exp(\imath \beta) = \exp(\imath \gamma) \in M_{\infty}
\begin{lemma}[Addition of Angles]
    \label{lem:construction_add_angle}
    For two given angle $\alpha$ (given by two streches $[a,b]$ and $[c,d]$) and $\beta$ (given by two streches $[e,f]$ and $[g,h]$), with $ a,b,c,d,e,f,g,h \in M_{\infty}$, the addition of the angles is constructable.
\end{lemma}
%TODO: add proof, and Construction Idea, 

%2.3.7
\begin{lemma}[multiplication of positive real numbers]
    \label{lem:construction_mult_pos_real}
    For $r_1, r_2 > 0 \in \mathbb{R}\cup M_{\infty}$ is $r_1 r_2 \in M_{\infty}$.
\end{lemma}
%TODO: add proof, and Construction Idea

%2.3.8
\begin{lemma}[Inverse of positive real numbers]
    \label{lem:construction_inv_pos_real}
    For $r > 0\in \mathbb{R}\cup M_{\infty}$ is $r^{-1} \in M_{\infty}$.
\end{lemma}
%TODO: add proof, and Construction Idea

%2.3.9
\begin{lemma}[Root of positive real numbers]
    \label{lem:construction_sqrt_pos_real}
    For $r > 0 \in \mathbb{R}\cup M_{\infty}$ is $\sqrt{r} \in M_{\infty}$.
\end{lemma}
%TODO: add proof, and Construction Idea

%2.3.10
\begin{lemma}[Polarkoordinates of complex numbers]
    \label{lem:construction_polar}
    For $z = r \exp(\imath \alpha) \in M_{\infty}$ is $r, \allowbreak \exp(\imath \alpha) \in M_{\infty}$.
\end{lemma}
%TODO: add proof, and Construction Idea

%2.3.11
\begin{lemma}[Real and Imaginary part of complex numbers]
    \label{lem:construction_re_im}
    For $0 \ne z = x + \imath y \in M_{\infty}$ is $x, \imath y \in M_{\infty}$.
\end{lemma}
%TODO: add proof, and Construction Idea

%2.3.12
\begin{lemma}[Construction of $\imath$]
    \label{lem:construction_imath}
    $\imath \in M_{\infty}$.
\end{lemma}
%TODO: add proof, and Construction Idea
