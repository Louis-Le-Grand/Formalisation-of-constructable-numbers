\chapter{Field of constructable Numbers}
\section[Filed of Constructable Numbers]{Field $M_{\infty}$}
%2.3
\begin{theorem}
    \label{thm:MField}
    \leanok
    \lean{MField}
    For $M\subseteq \mathbb{C}$ with $0,1 \in M$ is $M_{\infty}$ a subfield of $\mathbb{C}$.
\end{theorem}
\begin{proof}
    To show that $M_{\infty}$ is a subfield of $\mathbb{C}$ we have to show that $0,1\in M_{\infty}$ and $M_{\infty}$ is closed under addition, multiplication, subtraction and division.
    \begin{itemize}
        \item [$0,1$:] This follows from $0,1 \in M$ and Lemma \ref{lem:M_in_M_inf}.
        \item [$+$:] For $z_1,z_2 \in M_{\infty}$ we can construct $z_1 + z_2 \in M_{\infty}$. \ref{lem:construction_add}
        \item [$*$:] For $z_1,z_2 \in M_{\infty}$ we can construct $z_1 \cdot z_2 \in M_{\infty}$. \ref{cor:construction_mul_complex}
        \item [$-$:] For $z \in M_{\infty}$ we can construct $-z \in M_{\infty}$. \ref{lem:construction_neg}
        \item [$^{-1}$:] For $z \in M_{\infty}$ with $z \ne 0$ we can construct $z^{-1} \in M_{\infty}$. \ref{cor:inv_M_inf}
    \end{itemize}
\end{proof}

\begin{remark}
   To prove that $M_{\infty}$ is a subfield of $\mathbb{C}$ in Lean we have to creat a new instance with carrier $M_{\infty}$.
\end{remark}
\begin{remark}
    \label{rem:MField_Field}
    Since $M_{\infty}$ is a subfield of $\mathbb{C}$, $M_{\infty}$ is a field, wich is proof in Lean automatically.
\end{remark}


\begin{lemma}
    \label{lem:M_inf_properties}
    \leanok
    \lean{MField_i, MField_re_im, MField_re_im', MField_polar}
    \uses{thm:MField}
    For $M\subseteq \mathbb{C}$ with $0,1 \in M$ applies:
    \begin{itemize}
        \item[(i)] $\imath \in M_{\infty}$.
        \item[(ii)] For $z = x + \imath y \in \mathbb{C}$ is equivalent:
            \begin{itemize}
                \item $z \in M_{\infty}$.
                \item $x, y \in M_{\infty}$.
                \item $x, \imath y \in M_{\infty}$.
            \end{itemize}
        \item[(iii)] 
            For $0 \ne z = r \exp(\imath \alpha) \in \mathbb{C}$ is equivalent:
            \begin{itemize}
                \item $z \in M_{\infty}$.
                \item $r,\exp(\imath \alpha) \in M_{\infty}$.
            \end{itemize}
    \end{itemize}
\end{lemma}
\begin{proof}
    %TODO: \uses{} imath_M_inf z_iff_re_im_M_inf; (i_z_imp_z_in_M_inf/ MField_i_mul)
    This lemma is a dierct consequence of section \ref{basic_constructions}.
    \begin{itemize}
        \item[(i):] We kan apply construction \ref{cor:construction_imath}
        \item[(ii):] We can apply construction \ref{cor:z_iff_re_im_M_inf} and \ref{cor:construction_imath}.
        \item[(iii):]  We can apply construction \ref{lem:construction_radius} and \ref{lem:angle_M_inf}.
    \end{itemize}
\end{proof}

\subsection*{Quadritc closed}
\begin{definition}[quadritc closed field]
    \label{def:quadritc_closed_field}
    \leanok
    \lean{QuadraticClosed}
    A field $F$ is called \emph{quadritc closed} if for all $x \in F$ there is a $y \in F$ so that $y^2 = x$.
\end{definition}
\begin{remark}
    An equivalent definition is that $F$ is quadritc closed if $F=\{a^2\mid a \in K\}$.
\end{remark}

\begin{lemma}
    \label{lem:M_inf_quad_closed}
    \leanok
    \lean{MField_QuadraticClosed, MField_QuadraticClosed_def}
    \uses{def:set_of_constructable_points, def:quadritc_closed_field}
    For $M\subseteq \mathbb{C}$ with $0,1 \in M$ is $M_{\infty}$ quadritc closed.
\end{lemma}
\begin{proof}
    \uses{thm:MField, cor:root_M_inf}
    We know that $M_{\infty}$ is a field \ref{rem:MField_Field} and Lemma \ref{cor:root_M_inf} gives us for all $z \in M_{\infty}$ a $z^{\frac{1}{2}} \in M_{\infty}$.
    $$ z^{\frac{1}{2}} * z^{\frac{1}{2}} = z^{\frac{1}{2}^2} = z^{2\cdot \frac{1}{2}} = z.$$
    Therefore $M_{\infty}$ is quadritc closed.
\end{proof}

\subsection*{Conjugate closed}
\begin{definition}
    \label{def:conj_set}
    \leanok
    \lean{conj_set}
    For a Set $M \subset \mathbb{C}$ we define the \emph{conjugate set} of $M$ as 
    \begin{equation*}
        Conj(M) = \{\overline{z}\mid z\in M\}
    \end{equation*}
\end{definition}

\begin{lemma}
    \label{lem:conj_union}
    \leanok
    \lean{conj_union}
    For two set $M,N \subset \mathbb{C}$ $$Conj(M\cup N) = Conj(M) \cup Conj(N).$$
\end{lemma}
\begin{proof}
    For $z \in Conj(M\cup N)$ ther is a $w \in M\cup N$ so that $\overline{w} = z$, therefore $ z = \overline{w} \in Conj(M) \cup Conj(N)$. The other direction is analog.
\end{proof}

\begin{lemma}
    \label{lem:conj_conj_id}
    \leanok
    \lean{conj_conj_id}
    For a set $M \subset \mathbb{C}$ is $Conj(Conj(M)) = M$.
\end{lemma}
\begin{proof}
    $$Conj(Conj(M)) = Conj(\{\overline{z}\mid z\in M\}) = \{\overline{\overline{z}}\mid z\in M\} = \{ z \mid z\in M\} = M.$$
\end{proof}

\begin{definition}
    \label{def:conj_closed}
    \leanok
    \lean{ConjClosed}
    We call a subset of $\C$ \emph{conjugate closed} if $M= Conj(M)$.
\end{definition}


\begin{lemma}
    \label{lem:conj_MField}
    \leanok
    \lean{MField_conj}
    $M_{\infty}$ is conjugate closed.
\end{lemma}
\begin{proof}
    We can apply construction \ref{lem:construction_conj} and the fact that $\overline{\overline z} = z$ for all $z \in\mathbb{C}$.
\end{proof}

\begin{lemma}
    \label{lem:M_con_M}
    \leanok
    \lean{M_con_M}
    For $M\subseteq \mathbb{C}$ $M\cup Conj(M)$ is conjugate closed.
\end{lemma}
\begin{proof}
    We can apply Lemma \ref{lem:conj_union} and \ref{lem:conj_conj_id}.
    $$Conj(M\cup Conj(M)) = Conj(M) \cup Conj(Conj(M)) = M \cup Conj(M).$$
\end{proof}

\begin{lemma}
    \label{lem:ConjClosed.Rat_ConjClosed}
    \leanok
    \lean{ConjClosed.Rat_ConjClosed, ConjClosed.Rat_ConjClosed'}
    The set of rational numbers is conjugate closed.
\end{lemma}

\begin{proof}
    For every $r \in \mathbb{Q}$ we have $\overline{r} = r$.
\end{proof}

\begin{lemma}
    \label{lem:ConjClosed.conj_inclusion}
    \leanok
    \lean{ConjClosed.conj_inclusion}
    For $M,N \subseteq \mathbb{C}$ with $M \subseteq N$ is $Conj(M) \subseteq Conj(N)$.
\end{lemma}
\begin{proof}
    For $z \in Conj(M)$ there is a $w \in M$ so that $\overline{w} = z$ and since $M \subseteq N$ we have $w \in N$ and therefore $z \in Conj(N)$.
\end{proof}

\begin{lemma}
    \label{lem:ConjClosed.conj_field}
    \leanok
    \lean{ConjClosed.conj_field}
    For a subfield $F$  of $mathbb{C}$ the conjugate set $Conj(F)$ is a subfield of $\mathbb{C}$.
\end{lemma}
\begin{proof}
    We have to show that $0,1 \in Conj(F)$ and $Conj(F)$ is closed under addition, multiplication, subtraction and division.

    TODO: Dicede if it is worth to write out the proof.
\end{proof}

\begin{lemma}
    \label{lem:ConjClosed.re_im_in_L}
    \leanok
    \lean{ConjClosed.ir_L, ConjClosed.im_L}
    Let $L$ be a subfield of $\mathbb{C}$, with $L = conj(L)$. For all $z = x + \imath y \in L$ we have $x, \imath y \in L$.
\end{lemma}
\begin{proof}
    Let $z = x + \imath y \in L$. Since $L$ is conjugate closed we know that $\overline{z}=x-\imath y \in L$. This implies
    \begin{equation*}
        \frac{z + \overline{z}}{2} = x \in L
    \end{equation*}
    and therfore also $\imath y = z - x \in L$.
\end{proof}


\begin{lemma}
    \label{lem:ConjClosed.dist_sqard_in_L}
    \leanok
    \lean{ConjClosed.dist_sqard_in_L}
    Let $L$ be a subfield of $\mathbb{C}$, with $L = conj(L)$, and $z_1, z_2 \in L$.
    For $r := \|z_1-z_2\|$ we get that $r^2 \in L$.
\end{lemma}
\begin{proof}
    \uses{re_im_in_L}
    For $z_1 = x_1 + \imath y_1$ and $z_2 = x_2 + \imath y_2$ we have
    \begin{equation*}
        r = \|z_1 - z_2\| = \sqrt{(x_1 - x_2)^2 + (y_1 - y_2)^2}
    \end{equation*}
    and therefore
    \begin{equation*}
        r^2 = (x_1 - x_2)^2 + (y_1 - y_2)^2 
    \end{equation*}
    After applying Lemma \ref{lem:ConjClosed.re_im_in_L} we get $r^2 \in L$.
\end{proof}

\begin{lemma}
    \label{lem:ConjClosed.Intersection_line_line}
    \leanok
    \lean{ConjClosed.ill_L, ConjClosed.ill_L'}
    Let $L$ be a subfield of $\mathbb{C}$, with $L = conj(L)$. For $i = 1,2,3,4$ let $z_i = x_i + \imath y_i \in L$ with $z_1 \ne z_2$ and $z_3 \ne z_4$. Define
    %TODO: write in two lines
    \begin{equation*}\begin{aligned}
        G_1 := \{\lambda z_1 + (1-\lambda)z_2 \mid \lambda \in \mathbb{R}\},\\
        G_2 := \{\mu z_3 + (1-\mu)z_4 \mid \mu \in \mathbb{R}\}.
    \end{aligned} \end{equation*}
    If $G_1 \cap G_2 \ne \emptyset$ and $G_1 \ne G_2$. It is equivalent 
    \begin{itemize}
        \item $z\in G_1 \cap G_2$.
        \item Ther are $\lambda, \mu \in \mathbb{R}$ such that:
        \subitem - $\lambda(x_1 - x_2)+\mu(x_4-x_3) = x_4-x_2$
        \subitem - $\lambda(\imath y_1 - \imath y_2)+\mu(\imath y_4-\imath y_3) = \imath y_4-\imath y_2$
        \subitem - $z = \lambda z_1 + (1-\lambda)z_2$
    \end{itemize}
    In this case $z \in L$.
\end{lemma}
\begin{proof}
    \uses{re_im_in_L}
    The proof is done in two steps we show that $z \in G_1 \cap G_2$ if and only if there are $\lambda, \mu \in \mathbb{R}$ such that the equations hold. \\
    $\lambda(x_1 - x_2)+\mu(x_4-x_3) = x_4-x_2$ and $\lambda(\imath y_1 - \imath y_2)+\mu(\imath y_4-\imath y_3) = \imath y_4-\imath y_2$ 
    are equivalent to $\lambda x_1 + (1-\lambda)x_2 = \mu x_3 + (1-\mu)x_4$ and $\lambda y_1 + (1-\lambda)y_2 = \mu y_3 + (1-\mu)y_4$. Wich is the definition of $z \in G_1 \cap G_2$, split in real and imaginary part.\\
    The third equation is equivalent to $z = \lambda z_1 + (1-\lambda)z_2$ gives use that $z \in G_1$ at the piont wher $G_1$ and $G_2$ intersect, so we can assume that $z \in G_1 \cap G_2$.\\ \\
    Now we can show that $z \in L$.\\
    Since we now that z is equal to $\lambda z_1 + (1-\lambda)z_2$ and $z_1, z_2 \in L$ we only have to show that $\lambda \in L$. Here for we use the equations from the first part of the proof.
    \begin{align*}
        \RNum{1} &&\lambda(x_1 - x_2)+\mu(x_4-x_3) &= x_4-x_2\\
        \RNum{2} &&\lambda(\imath y_1 - \imath y_2)+\mu(\imath y_4-\imath y_3) &= \imath y_4-\imath y_2
    \end{align*}
    Now we solve \RNum{2} for $\mu$ 
    \begin{align*}
        && \lambda(\imath y_1 - \imath y_2)+\mu(\imath y_4-\imath y_3) &= \imath y_4-\imath y_2 && \mid -\lambda(\imath y_1 - \imath y_2)\\
        &\Leftrightarrow & \mu(\imath y_4-\imath y_3) &= \imath y_4-\imath y_2 - \lambda(\imath y_1 - \imath y_2) && \mid \div \imath(y_4-y_3)\\
        &\Leftrightarrow & \mu &= \frac{\imath y_4-\imath y_2 - \lambda(\imath y_1 - \imath y_2)}{\imath y_4-\imath y_3}\\
       %&\Leftrightarrow & \mu(y_4-y_3) &= y_4-y_2 - \lambda(y_1 - y_2) &&\mid \div \imath(y_4-y_3)\\
        %&\Leftrightarrow & \mu &= \frac{y_4-y_2 - \lambda(y_1 - y_2)}{y_4-y_3}\\
    \end{align*}
    Since we divided by $\imath (y_4-y_3)$ we need to assume that $y_4 \ne y_3$, so we need to first handel the case $y_4 = y_3$.\\
    If $y_4 = y_3$ we have $\lambda(\imath y_1 - \imath y_2) = \imath y_4-\imath y_2$ and since $y_4 = y_3$ $y_1 \ne y_2$, because otherwise boothe Lines would be parrale to the Rael line an therefore $G_1 = G_2$ or $G_1 \cap G_2 = \varnothing$. Therefore $\lambda = \frac{\imath y_4-\imath y_2}{\imath y_1 - \imath y_2}$. Using the fact that real part and the imaginary part times $\imath$ are in $L$ \ref{re_im_in_L} we have written $\lambda$ as a fraction of two elements in $L$. Therefore $\lambda$ is in $L$, wich implies that $z = \lambda z_1 + (1-\lambda)z_2$ is in $L$.\\
   
    Now we insert $\mu$ in \RNum{1} and solve for $\lambda$.\\
    \resizebox{1\linewidth}{!}{
    \begin{minipage}{\linewidth}
    \begin{alignat*}{3}
        && \lambda(x_1 - x_2)+\mu(x_4-x_3) &= x_4-x_2 && \mid \RNum{1} \leftarrow \RNum{2}\\
        &\Leftrightarrow & \lambda(x_1 - x_2)+\frac{\imath y_4-\imath y_2 - \lambda(\imath y_1 - \imath y_2)}{\imath y_4-\imath y_3}(x_4-x_3) &= x_4-x_2 && \mid \cdot (\imath y_4-\imath y_3)\\
        &\Leftrightarrow & \lambda(x_1 - x_2)(\imath y_4-\imath y_3)+(\imath y_4-\imath y_2 - \lambda(\imath y_1 - \imath y_2))(x_4-x_3) &= (x_4-x_2)(\imath y_4-\imath y_3) && \mid - (x_1 - x_2)(\imath y_4-\imath y_3)\\
        %&\Leftrightarrow & \lambda(y_4-y_3)(x_1 - x_2)+(y_4-y_2)(x_4-x_3) - \lambda(y_1 - y_2)(x_4-x_3) &= (x_4-x_2)(y_4-y_3) && 
        &\Leftrightarrow & \lambda((x_1 - x_2)(\imath y_4-\imath y_3) - (\imath y_4-\imath y_2)(x_4-x_3)) &= (x_4-x_2)(\imath y_4-\imath y_3) - (\imath y_4-\imath y_2)(x_4-x_3) && \mid \div ((\imath y_4-\imath y_3)(x_1 - x_2)- (\imath y_1 - \imath y_2)(x_4-x_3))\\
        %&\Leftrightarrow & \lambda((y_4-y_3)(x_1 - x_2)- (y_1 - y_2)(x_4-x_3)) &= (x_4-x_2)(y_4-y_3) - (y_4-y_2)(x_4-x_3) && \mid \div ((y_4-y_3)(x_1 - x_2)- (y_1 - y_2)(x_4-x_3))\\
        &\Leftrightarrow & \lambda &= \frac{(x_4-x_2)(\imath y_4-\imath y_3) - (\imath y_4-\imath y_2)(x_4-x_3)}{(\imath y_4-\imath y_3)(x_1 - x_2)- (\imath y_1 - \imath y_2)(x_4-x_3)}\\
    \end{alignat*}
    \end{minipage}
    }\\
    We need to check that the denominator $(y_4-y_3)(x_1 - x_2)- (y_1 - y_2)(x_4-x_3)$ is not zero.
    If would be zero we would have $(y_4-y_3)(x_1 - x_2) = (y_1 - y_2)(x_4-x_3)$, wich is equivalent to $\frac{y_4-y_3}{x_4-x_3} = \frac{y_1 - y_2}{x_1 - x_2}$. This would mean that the two lines are parralel and therefore $G_1 = G_2$ or $G_1 \cup G_2 = \varnothing$.\\
    So we can assume that the denominator is not zero and therefore we can write $\lambda$ as a fraction of two elements in $L$. Therefore $\lambda$ is in $L$, wich implies that $z = \lambda z_1 + (1-\lambda)z_2$ is in $L$.

\end{proof}

%3.17
\begin{lemma}
    \label{Intersection_line_circle}
    Let $L$ be a subfield of $\mathbb{C}$, with $L = conj(L)$. For $i = 1,2,3$ let $z_i = x_i + \imath y_i \in L$ with $z_1 \ne z_2$, and let $r > 0$ in $\mathbb{R}$ with $r^2 \in L$. Define
    \begin{equation*}\begin{aligned}
        G := \{\lambda z_1 + (1-\lambda)z_2 \mid \lambda \in \mathbb{R}\},\\
        C := \{z = x + \imath y \in \mathbb{C} \mid \|z - z_3\| = r\}.
    \end{aligned} \end{equation*}
    If $G \cap C \ne \emptyset$. Then the following is equivalent:
    \begin{itemize}
        \item $z\in G \cap C$.
        \item There is a $\lambda \in \mathbb{R}$ with $\lambda^2 a+ \lambda b + c = 0$ where
        \begin{align*}
            a &:= (x_1 - x_2)^2 + (\imath y_1 - \imath y_2)^2,\\
            b &:= 2(x_1 - x_2)(x_2 - x_3) - 2(\imath y_1 - \imath y_2)(\imath y_2 - \imath y_3),\\
            c &:= (x_2 - x_3)^2 + (\imath y_2 - \imath y_3)^2 - r^2,
        \end{align*}
        and $z = \lambda z_1 + (1-\lambda)z_2$.
    \end{itemize}
    In this case $z \in L(\sqrt{w})$ for an $w \in L$.
\end{lemma}

\begin{proof}
First we have to show $z \in G \cap C$ iff and only iff ther existe a $\lambda \in \mathbb{R}$ with $\lambda^2 a+ \lambda b + c = 0$ and $z = \lambda z_1 + (1-\lambda)z_2$.

TODO Umstellen \\
Now we can show that there existe a $w \in L$ such that $z \in L(\sqrt{w})$.\\
Since we know that $z = \lambda z_1 + (1-\lambda)z_2$ and $z_1, z_2 \in L$ we only have to show that $\lambda \in L(\sqrt{w})$. Here for we use the equations from the first part of the proof. 
Since Lambda is a solution of a quadratic equation we now that $\lambda$ is equal to $\frac{-b \pm \sqrt{b^2 - 4ac}}{2a}$. Since $a,b,c \in L$ we get $w = b^2 - 4ac \in L$ so $\lambda \in L(\sqrt{w})$. Therefore $z = \lambda z_1 + (1-\lambda)z_2$ is in $L(\sqrt{w})$.
%TODO: add proof
\end{proof}

%3.18
\begin{lemma}
    \label{Intersection_circle_circle}
    Let $L$ be a subfield of $\mathbb{C}$, with $L = conj(L)$. For $i = 1,2 $ let $z_i = x_i + \imath y_i \in L$ with $z_1 \ne z_2$ and let $r_i > 0$ in $\mathbb{R}$ with $r_i^2 \in L$. Define
    \begin{equation*} \begin{aligned}
        C_1 := \{z = x + \imath y \in \mathbb{C} \mid \|z - z_1\| = r_1\},\\
        C_2 := \{z = x + \imath y \in \mathbb{C} \mid \|z - z_2\| = r_2\}.
    \end{aligned} \end{equation*}
    If $C_1 \cap C_2 \ne \emptyset$ and $C_1 \ne C_2$. Then 
    $$G := \{x+\imath y \in \mathbb{C} \mid 2(x_2 - x_1)x - 2(\imath y_2 - \imath y_1)\imath y = r_1^2 - r_2^2 + x_2^2 - x_1^2 + (\imath y_2)^2 - (\imath y_1)^2\} $$
    is a real line, and $$ C_1 \cap C_2 = G \cap C_1 = G \cap C_2. $$
    For $z \in C_1 \cap C_2$ there is a $w \in L$ such that $z \in L(\sqrt{w})$.
\end{lemma}
\begin{proof}
    Sorry,  like in lean %TODO: add proof
\end{proof}

\section[K zero]{The Field $\mathcal{K}_0$}
\begin{definition}
    \label{def:K_M_0}
    Let $\mathcal(M)\subseteq\mathbb{C}$ with $0,1 \in \mathcal{M}$
    \begin{equation*}
        K_0 := \mathbb{Q}(\mathcal{M}\cup Conj(\mathcal{M}))
    \end{equation*}
\end{definition}

\begin{lemma}
    \label{lem:conj_adjion}
    \leanok
    \lean{conj_adjoin, conj_adjion'}
    Let $K$ be an conjugation closed intermidiate field of $\Q$ $\C$ and $ M \subset \mathbb{C}$ be a subset with $M = conj(M)$. Then holds
    $K(M)$ is conjugate closed.
\end{lemma}
\begin{proof}
    In reference \ref{lem:ConjClosed.conj_field}, it was demonstrated that for a field F, the field of complex numbers, $Conj(F)$ is a field. 
    It can thus be concluded that $Conj(K(M))$ is also a field.
    As both $K$ and $M$ are subsets of $K(M)$, it can be inferred from lemma \ref{lem:ConjClosed.conj_inclusion} that $Conj(K) \overset{ConjClosed}{=} K$ and $Conj(M) \overset{ConjClosed}{=} M$ are subsets of $Conj(K(M))$. 
    As $K(M)$ is the smallest subfield of $\C$ that includes it, it can be concluded that $$K(M) \subseteq Conj(K(M)).$$
    Furthermore, if we apply $Conj$ to both sides and again infer \ref{lem:ConjClosed.conj_inclusion}, we obtain $$Conj(K(M)) \subseteq Conj(Conj(K(M))) = K(M),$$ 
    which leads to the conclusion that $Conj(K(M)) = K(M)$.
\end{proof}

\begin{corollary}
    \label{cor:K_zero_conjectClosed}
    \leanok
    \lean{K_zero_conjectClosed}
    For $M\subseteq \mathbb{C}$ with $0,1 \in M$ is $K_0$ conjugate closed.
\end{corollary}
\begin{proof}
    By employing the preceding lemma, it is sufficient to demonstrate that the $\Q$ and $M \cup Conj(M)$ are conjugate closed. 
    Wich can be infered from \ref{lem:ConjClosed.Rat_ConjClosed} and \ref{lem:M_con_M}
\end{proof}

\begin{lemma}
    \label{lem:K_zero_in_MField}
    \leanok
    \lean{K_zero_in_MField}
    For $M\subseteq \mathbb{C}$ with $0,1 \in M$ is $K_0 \subseteq M_{\infty}$.
\end{lemma}
\begin{proof}
From the definition of $K_0 := \Q(M\cup Conj(M))$, 
it can be seen that this is the smallest subfield of $\C$ containing both $Q$ and $M\cup Conj(M)$. 
Consequently, it is sufficient to demonstrate that both $\Q$ and $M\cup Conj(M)$ are contained within $M_{\infty}$. 
Since $\Q$ is contained in every subfield of $\C$, it is therefore also contained in $M_{\infty}$. 
Furthermore, since $M$ is contained in $M_{\infty}$ \ref{lem:M_in_M_inf} and $M$ is conjugate closed \ref{lem:conj_MField}, we can conclude that $M\cup Conj(M) \subseteq \Q$.
\end{proof}


\section{Joshuas Kapitel}

\begin{definition}
    The degree of $x$ over $K$ is
    \begin{equation*}
        [x:K] :=\text{degree}(\mu_{x,K})
        \end{equation*}
        with $\mu_{x,K}$ the minimal polynomial of $x$ over $K$. \newline
    The degree of $L/K$ is the dimension of $L$ as a $K$-vector space and is denoted by
    \begin{equation*}
        [L:K].
    \end{equation*}
\end{definition}

\begin{theorem}
\label{thm:degree_of_simple_field_extension}
    Let $L/K$ be a simple field extension with $L = K(x)$. Then
    \begin{equation*}
        [L:K] = [x:K].
    \end{equation*}
\end{theorem}
\begin{proof}
    In Mathlib: theorem IntermediateField.adjoin.finrank
\end{proof}

\begin{definition}
    %\label{def:K_M_0}
    Let $\mathcal(M)\subseteq\mathbb{C}$ with $0,1 \in \mathcal{M}$
    \begin{equation*}
        K_0 := \mathbb{Q}(\mathcal{M}\cup \overline{\mathcal{M}})
    \end{equation*}
    with $\overline{\mathcal{M}} := \{ \overline{z} = x - \textbf{i}y \mid z = x+\textbf{i}y  \in \mathcal{M} \}$.
\end{definition}

TODO: Add $K_i$ and $K_{\infty}$ and a Lemma that states that $K_{\infty} = M_{\infty}$. But this is not needed for the project.

%3.20
\begin{lemma}
    \label{lem:conj_of_adjoin}
    Let $K$ be a subfield of $\mathbb{C}$ with $conj(K)=K$, and $ M \subset \mathbb{C}$ be a subset with $M = conj(M)$. Then holds
    \begin{equation*}
        conj(K(M)) = K(M).
    \end{equation*}
\end{lemma}
\begin{proof}
    We know that the complex conjugation is a field automorphism $\overline{\cdot }: \mathbb{C} \to \mathbb{C}$, whit $\overline{\overline{z}} = z$ for all $z \in \mathbb{C}$. Therefore we have
    $conj(K(M))$ is a subfield of $\mathbb{C}$.\\
    Since $K$ and $M$ are subsets of $K(m)$ we have $conj(K) = K$ and $conj(M) = M$ are subsets of $conj(K(M))$. Therefore $$K(M) \subset conj(K(M)).$$
    If we know apply $conj$ to both sides we get
    $$conj(K(M)) \subset conj(conj(K(M))) = K(M).$$
    Therefore $conj(K(M)) = K(M)$.
\end{proof}

%3.19
\begin{lemma}
    \label{lem:dergee2_iff_adjoint_sqrt}
    Let $K\subset L \subset \mathbb{C}$ be fields. Then the following is equivalent:
    \begin{itemize}
        \item $[L:K] = 2$.
        \item There is a $w \in K$ with $\sqrt{w} \notin K$ and $L = K(\sqrt{w})$.
    \end{itemize}
\end{lemma}
\begin{proof}
    $(ii)\implies (i)$: Let $w$ be as in $(ii)$.Then $\sqrt{w}$ is a root of $X^2 - w \in K[X]$. Since $\sqrt{w} \notin K$ this polynomial is irreducible in $K[X]$. Therefore $[L:K] = 2$.\\
    $(i)\implies (ii)$: Let $[L:K] = 2$ and $a \in L \setminus K$. Then $K(\alpha) = L$ and 
    $$\mu_{\alpha, K}=X^2 + bX + c \quad b,c \in K$$
    This implies 
    $$\alpha = -\frac{b}{2} \pm \sqrt{\frac{b^2}{4} - c} $$
    Now let $w := \frac{b^2}{4} - c \in K$ then we get $L = K(\alpha) = K(\sqrt{w})$.
\end{proof}


%TODO: add lemma wich $K_0 \subset M_{\infty}$

%3.21
\begin{theorem}[constructable iff chain dergee2]
    \label{thm:Z_in_Minf_1}
    For $z \in \mathbb{C}$ is equivalent:
    \begin{itemize}
        \item $z \in M_{\infty}$
        \item There is an intermidiate field $L$ of $\mathbb{C}/K_0$ whit $z \in L$, so that $L$ aries from $K_0$ by a sequence of adjunctions of square roots.
        \item There is an $n\le 0$ and chain 
            $$K_0 = L_1 \subset L_2 \subset \ldots \subset L_n \subset \mathbb{C}$$
            of subfields of $\mathbb{C}$ so that $z \in L_n$ and 
            $$ [L_{i+1}:L_i] = 2 \quad \text{for} \quad i = 0, \ldots, n-1.$$
    \end{itemize}
\end{theorem}

\begin{remark}
    \label{rem:Z_in_Minf_1}
    In this cases the $[K_0(z):K_0] = 2^m$ for some $0 \le m \le n$.
\end{remark}
\begin{proof}[Proof of Theorem \ref{thm:Z_in_Minf_1}]
    \uses{ lem:dergee2_iff_adjoint_sqrt, Intersection_circle_circle, Intersection_line_circle, Intersection_line_line, lem:conj_of_adjoin, lem:M_inf_quad_closed}
    $(ii) \iff (iii)$ Follows from the definition and Lemma \ref{lem:dergee2_iff_adjoint_sqrt}.\\
    $(i) \implies (ii)$: We assume that $z$ is derived from $K_0$ by (ZL1), (ZL2) and (ZL3). %TODO: add ref
    Lemma \ref{Intersection_circle_circle}, \ref{Intersection_line_circle} and \ref{Intersection_line_line} in combination with Lemma \ref{lem:conj_of_adjoin} 
    imply that ther is an $w \in K_0$ so that $z \in K_0(\sqrt{w})$. Since $\overline{w} \in K_0$, by definition. We define
    $$K_1 := K_0(\sqrt{w},\sqrt{\overline{w}}).$$
    $K_1$ derives from $K_0$ sucsessively by adjunction of square roots. Note that $K_1 = conj(K_1)$, by Lemma \ref{lem:conj_of_adjoin} and the fact that $\sqrt{\overline{w}} = \overline{\sqrt{w}}$.
    Because of the definition of $$M_{\infty} = \bigcup_{n=0}^{\infty} M_n,$$
    he statement follows by induction.\\ %TODO: Elaborate on induction for lean
    $(ii) \implies (i)$: We know that $M_{\infty}$ is an quadritc complet field \ref{lem:M_inf_quad_closed}, i.e. $\sqrt{z} \in M_{\infty}$ for all $z \in M_{\infty}$.
    Futhermore we know that $K_0\subseteq M_{\infty} \subseteq \mathbb{C}$, %TODO: add ref
    therefore every field $L\subseteq \mathbb{C}$ which is derived from $K_0$ by a sequence of adjunctions of square roots is contaianted in $M_{\infty}$.
\end{proof}